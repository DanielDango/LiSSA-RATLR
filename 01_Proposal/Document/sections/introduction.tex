%% LaTeX2e class for student theses
%% sections/content.tex
%% 
%% Karlsruhe Institute of Technology
%% Institute for Program Structures and Data Organization
%% Chair for Software Design and Quality (SDQ)
%%
%% Dr.-Ing. Erik Burger
%% burger@kit.edu
%%
%% Version 1.6, 2024-06-07

\chapter{Introduction}
\label{ch:Introduction}

Artificial intelligence is rapidly changing many fields of study.
During software development, many artifacts are created, ranging from the actual project code through documentation to a multitude of formal and informal diagrams. My goal is to improve the rate at which trace links are recovered using prompt engineering.

\section{Definition of Trace Link Recovery}
Traceability is the ability for something to be traced. Meaning, there is evidence of some past occurrence. %Do I need to quote a dictionary here?
The task of trace link recovery (TLR) in software engineering is to find instances of the same element across different artifacts and link them for further processing. These traceability links help with tracking relationships between for example: code, requirements, diagrams, documentation, \dots

This becomes especially important when inconsistencies are introduced into the project's artifacts. 

As shown by \citeauthor{wohlrab2019improving} \cite{wohlrab2019improving} inconsistency in wording and language is quite common during different stages of development. For example, naming conventions for architectural components may not be followed during implementation. They find the impact of these inconsistencies to be quite insignificant. However, for trace link recovery, this means that simple string comparisons by name are insufficient.



\section{Definition of Automated Prompt Engineering}
Prompt engineering is the process of refining a prompt for the specific use case. This is usually done in a non-systematic way. In this work, I will explore a systematic iterative approach to prompt engineering.

\section{Previous Work in TLR}
In previous work by \citewithauthor{fuchss2025lissa}
simple handwritten prompts were used to recover trace links between software documentation and architectural diagrams. They have made their work available to the public by hosting a repository, accessible under the MIT license. 