Question: Here are two parts of software development artifacts.\newline
\newline
\{source\_type\}: \textquotesingle{}\textquotesingle{}\textquotesingle{}\{source\_content\}\textquotesingle{}\textquotesingle{}\textquotesingle{}\newline
\newline
\{target\_type\}: \textquotesingle{}\textquotesingle{}\textquotesingle{}\{target\_content\}\textquotesingle{}\textquotesingle{}\textquotesingle{}\newline
Are they related?\newline
\newline
Answer with \textquotesingle{}yes\textquotesingle{} or \textquotesingle{}no\textquotesingle{}.\newline
\newline
Class Explanations:\newline
\newline
1. \*\*Yes\*\*: This classification is used when the two artifacts are directly related or connected in a meaningful way. For example, if one requirement specifies a feature or functionality that is directly supported or enabled by another requirement, they should be classified as \textquotesingle{}Yes\textquotesingle{}. In the provided examples, if a requirement specifies the need for a universal interface to create WARC records and another requirement details the types of WARC records that can be created through such an interface, they are related and should be classified as \textquotesingle{}Yes\textquotesingle{}. This implies a direct relationship where one requirement\textquotesingle{}s implementation or purpose is explicitly supported or fulfilled by the other. For instance, if "FR 3" specifies providing functions through a universal interface for creating WARC records, and "SRS 7" specifies the types of WARC records that can be created through such an interface, they are directly related. This direct relationship is characterized by one requirement enabling or detailing the implementation of another, ensuring that the functionality or purpose is clearly aligned and supported.\newline
\newline
2. \*\*No\*\*: This classification is used when the two artifacts are not directly related or do not have a meaningful connection. For instance, if one requirement discusses the encapsulation of internal functionality and another requirement focuses on ensuring compatibility between versions, they address different aspects and should be classified as \textquotesingle{}No\textquotesingle{}. In the examples provided, if one requirement specifies the need for a single header file and another requirement details the types of WARC records, they are not directly related and should be classified as \textquotesingle{}No\textquotesingle{}. This indicates that the requirements do not share a direct or explicit connection in terms of functionality or purpose, even if they are part of the same broader system or project. For example, "FR 1" requiring a single header file and "SRS 7" detailing WARC record types do not directly relate, as they address different aspects of the system. The lack of a direct relationship is evident when the requirements focus on separate functionalities or objectives, without one directly supporting or fulfilling the other.\newline
\newline
Incorporate these explanations to ensure accurate classification by identifying direct relationships or lack thereof between the requirements.