<prompt\>\newline
\textquotesingle{}\textquotesingle{}\textquotesingle{}Question: Here are two parts of software development artifacts.\newline
\newline
\{source\_type\}: \textquotesingle{}\textquotesingle{}\textquotesingle{}\{source\_content\}\textquotesingle{}\textquotesingle{}\textquotesingle{}\newline
\newline
\{target\_type\}: \textquotesingle{}\textquotesingle{}\textquotesingle{}\{target\_content\}\textquotesingle{}\textquotesingle{}\textquotesingle{}\newline
Are they related?\newline
\newline
Answer with \textquotesingle{}yes\textquotesingle{} or \textquotesingle{}no\textquotesingle{}.\newline
\newline
\*\*Class Explanations:\*\*\newline
\newline
1. \*\*Related Requirements\*\*: This category includes instances where the source and target requirements directly support or complement each other. For example, if the source requirement states that "Any software tool or application based on libwarc shall require just a single header file," and the target requirement specifies that "The \textquotesingle{}libwarc\textquotesingle{} headers shall be structured in a hierarchical manner," these are not related because the second requirement introduces a complexity that contradicts the simplicity implied by the first. In this case, the first requirement suggests a straightforward implementation, while the second implies a more complex structure that could lead to confusion about the intended use of the header file. Thus, if the requirements diverge in their implications or introduce conflicting concepts, they should be classified as unrelated.\newline
\newline
2. \*\*Unrelated Requirements\*\*: This category encompasses cases where the source and target requirements do not align in purpose or functionality. For instance, if the source requirement emphasizes a single entry point to a library, while the target requirement discusses the ability to access and manipulate all aspects of a file format through that entry point, they may seem related at a glance but actually serve different objectives. The first focuses on simplicity, while the second implies a broader functionality that could lead to confusion. An example of this would be the source stating a need for a single header file, while the target discusses multiple functionalities that could complicate the use of that header. Therefore, if the requirements serve distinct purposes or introduce different levels of complexity, they should be classified as unrelated.\newline
\newline
3. \*\*Complementary Requirements\*\*: This category is for requirements that, while distinct, enhance the overall functionality when considered together. For example, if one requirement states the need for a single header file and another discusses the accessibility of the WARC file format through that header, they could be seen as complementary if they both contribute to a unified goal of ease of use and functionality. However, if the second requirement introduces additional complexity or contradicts the simplicity of the first, they may not be complementary after all. It is essential to evaluate whether the requirements work together to achieve a common goal or if they create confusion through conflicting details.\newline
\newline
\*\*Note\*\*: The classification results in the provided examples indicate that the requirements are not related, as they either introduce conflicting concepts or serve different purposes. It is crucial to analyze the intent and implications of each requirement to determine their relationship accurately. Misclassifications can occur when the nuances of simplicity versus complexity are not fully considered, as seen in the examples provided. Always ensure to assess the relationship based on the clarity and intent of the requirements involved.