Question: Here are two parts of software development artifacts.\newline
\newline
\{source\_type\}: \textquotesingle{}\textquotesingle{}\textquotesingle{}\{source\_content\}\textquotesingle{}\textquotesingle{}\textquotesingle{}\newline
\newline
\{target\_type\}: \textquotesingle{}\textquotesingle{}\textquotesingle{}\{target\_content\}\textquotesingle{}\textquotesingle{}\textquotesingle{}\newline
Are they related?\newline
\newline
Answer with \textquotesingle{}yes\textquotesingle{} or \textquotesingle{}no\textquotesingle{}.\newline
\newline
Class Explanations:\newline
\newline
1. \*\*Yes\*\*: The two artifacts are related if they refer to the same concept, functionality, or requirement. For example, if both artifacts mention a single header file required for a software tool or application based on libwarc, they are related. This includes scenarios where one artifact describes the necessity of a single entry point, and the other explains the structure or functionality enabled by this single header file.\newline
\newline
2. \*\*No\*\*: The two artifacts are not related if they discuss different concepts, functionalities, or requirements. For instance, if one artifact talks about a single header file requirement and the other discusses a different aspect of the software that does not directly relate to the single header file, they are not related. However, if the second artifact implicitly supports or expands on the concept of a single header file, such as by detailing its structure or capabilities, they should be considered related.\newline
\newline
Examples:\newline
\newline
\- If a requirement states that a single header file is needed, and another requirement specifies that this header file includes all necessary components, they are related.\newline
\- If a requirement mentions a single entry point and another requirement discusses the hierarchical structure of headers that the entry point encompasses, they are related.\newline
\- If a requirement specifies the ability to manipulate aspects of a file format through a single header file, and another requirement confirms this capability, they are related.\newline
\newline
Use these examples to guide your classification decisions, ensuring that implicit connections are recognized and appropriately classified as \textquotesingle{}yes\textquotesingle{}.