Question: Here are two parts of software development artifacts.\newline
\newline
\{source\_type\}: \textquotesingle{}\textquotesingle{}\textquotesingle{}\{source\_content\}\textquotesingle{}\textquotesingle{}\textquotesingle{}\newline
\newline
\{target\_type\}: \textquotesingle{}\textquotesingle{}\textquotesingle{}\{target\_content\}\textquotesingle{}\textquotesingle{}\textquotesingle{}\newline
Are they related?\newline
\newline
Answer with \textquotesingle{}yes\textquotesingle{} or \textquotesingle{}no\textquotesingle{}.\newline
\newline
\*\*Class Explanations:\*\*\newline
\newline
1. \*\*Related\*\*: This category indicates that the two artifacts are connected in a meaningful way. For example, if the source requirement states that "Any software tool or application based on libwarc shall require just a single header file," and the target specification states, "There shall be a single entry point to libwarc, called \textquotesingle{}warc.h\textquotesingle{}," these two statements are related because they both address the structure and requirements of the libwarc library. The requirement implies the need for a single header file, and the specification identifies what that header file is. This connection shows that both statements contribute to the same aspect of the software\textquotesingle{}s architecture, thus establishing a clear relationship. In essence, if one statement provides a requirement that is fulfilled or specified by the other, they are considered related.\newline
\newline
2. \*\*Not Related\*\*: This category signifies that the two artifacts do not share a connection or relevance. For instance, if the source requirement discusses a feature unrelated to the target specification, such as "The application must support multiple languages," while the target specification focuses solely on a technical detail like "The application shall use a specific database," these two statements are not related. They address different aspects of the software development process, with one focusing on user interface capabilities and the other on backend infrastructure. This lack of overlap in context indicates that there is no meaningful connection between the two artifacts. In cases where the statements pertain to entirely different functionalities or requirements, they should be classified as not related.\newline
\newline
\*\*Guidelines for Classification\*\*:\newline
\- Carefully analyze the content of both the source and target artifacts.\newline
\- Look for keywords and phrases that indicate a direct relationship or relevance, such as terms that refer to the same component, feature, or requirement.\newline
\- Consider the context of each statement to determine if they address the same feature, requirement, or specification. Pay attention to the implications of each statement and how they may relate to the overall functionality or architecture of the software.\newline
\- If the source requirement outlines a condition that is directly addressed or specified by the target artifact, classify them as related. Conversely, if they discuss different aspects or functionalities without any overlap, classify them as not related.\newline
\newline
By following these guidelines and understanding the class explanations, you can make a more informed decision on whether the two artifacts are related or not. This will help ensure accurate classification based on the meaningful connections or lack thereof between the software development artifacts.