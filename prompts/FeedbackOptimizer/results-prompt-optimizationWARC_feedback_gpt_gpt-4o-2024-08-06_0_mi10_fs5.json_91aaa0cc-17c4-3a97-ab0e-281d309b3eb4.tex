Question: Here are two parts of software development artifacts.\newline
\newline
\{source\_type\}: \textquotesingle{}\textquotesingle{}\textquotesingle{}\{source\_content\}\textquotesingle{}\textquotesingle{}\textquotesingle{}\newline
\newline
\{target\_type\}: \textquotesingle{}\textquotesingle{}\textquotesingle{}\{target\_content\}\textquotesingle{}\textquotesingle{}\textquotesingle{}\newline
\newline
Consider the following when determining if they are related:\newline
1. Look for shared terminology or concepts, such as specific file names, structures, or functionalities. Pay attention to terms like "single header file," "universal header," and specific file names like "warc.h."\newline
2. Identify if one requirement directly supports, implements, or complements the functionality described in the other. Consider if one requirement describes a feature that is necessary for the implementation of the other.\newline
3. Determine if there is a logical connection, dependency, or enhancement between the two requirements. Consider if one requirement ensures compatibility, access, or encapsulation that is relevant to the other.\newline
4. Evaluate if the requirements address the same aspect of the system, such as compatibility, accessibility, or structure, even if they use different wording.\newline
5. Consider the intent and purpose behind each requirement. Determine if they share a common goal or outcome, such as simplifying integration, ensuring compatibility, or providing comprehensive access.\newline
6. Assess if the requirements are part of a sequence or workflow that contributes to a larger functionality or system behavior, indicating a traceability link.\newline
7. Pay attention to the context and implications of each requirement. Consider if they are part of a broader architectural or design principle that connects them.\newline
8. Look for implicit relationships where one requirement might imply the necessity of the other, even if not explicitly stated.\newline
9. Consider the broader impact of each requirement on the system and whether they collectively contribute to a cohesive design or functionality.\newline
10. Examine if the requirements are addressing the same problem or solution space, even if they approach it from different angles or levels of detail.\newline
11. Analyze if the requirements are interdependent, where the fulfillment of one is contingent upon the other, indicating a strong traceability link.\newline
12. Consider if the requirements are describing different aspects of the same feature or functionality, which might indicate a traceability link.\newline
13. Look for any mention of ensuring or maintaining a specific feature or functionality that is described in the other requirement.\newline
14. Evaluate if the requirements are part of a common theme or objective, such as improving usability, performance, or maintainability, which might suggest a relationship.\newline
15. Consider if the requirements are addressing the same technical constraints or design considerations, which could indicate a relationship.\newline
16. Look for any shared assumptions or preconditions that might link the requirements together.\newline
17. Determine if the requirements are part of a modular design where one module\textquotesingle{}s functionality is dependent on or enhanced by the other.\newline
18. Consider if the requirements are part of a layered architecture where one layer\textquotesingle{}s requirements are inherently linked to another layer\textquotesingle{}s requirements.\newline
19. Pay special attention to requirements that mention compatibility, encapsulation, or abstraction, as these often indicate a relationship with other requirements focusing on integration or interface design.\newline
20. Look for requirements that mention a single point of access or entry, as these are often related to requirements ensuring simplicity and ease of use.\newline
21. Consider if the requirements are part of a hierarchical structure where one requirement\textquotesingle{}s fulfillment is necessary for the other.\newline
22. Look for any indication that one requirement\textquotesingle{}s implementation or design is dependent on the other, suggesting a traceability link.\newline
23. Evaluate if the requirements collectively contribute to a specific design pattern or architectural style, indicating a relationship.\newline
24. Consider if the requirements are addressing different layers of abstraction within the same system, which might suggest a connection.\newline
25. Look for any shared goals or outcomes that are explicitly or implicitly stated, indicating a potential relationship.\newline
26. Pay attention to the specific phrasing and context of each requirement to identify nuanced relationships that may not be immediately obvious.\newline
27. Consider if the requirements are part of a broader system goal or objective that inherently links them together.\newline
28. Look for any historical or versioning context that might suggest a relationship, such as references to maintaining backward compatibility or evolving a feature.\newline
29. Evaluate if the requirements are part of a strategic initiative or roadmap that aligns them towards a common future state or capability.\newline
30. Consider if the requirements are addressing different user needs or scenarios that are interconnected, suggesting a relationship.\newline
\newline
Are they related?\newline
\newline
Answer with \textquotesingle{}yes\textquotesingle{} or \textquotesingle{}no\textquotesingle{}.