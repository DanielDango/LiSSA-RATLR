%% LaTeX2e class for student theses
%% sections/evaluation.tex
%% 
%% Karlsruhe Institute of Technology
%% Institute of Information Security and Dependability
%% Software Design and Quality (SDQ)
%%
%% Dr.-Ing. Erik Burger
%% burger@kit.edu
%%
%% Version 1.6, 2024-06-07

\chapter{Evaluation}
\label{ch:Evaluation}

\Todo{Talk about evaluation in general}


\section{Setup}
\label{sec:Evaluation:setup}
To evaluate the \APE algorithms proposed in \Todo{add link to approach}, a multitude of different datasets will be used.
They are taken from \citewithauthor{hey2025ReplicationPackage}'s replication package.
Some of the gold standard files were modified to provide consistency between artifact naming and the gold standard reference.
The actual contents are not affected by this.
The CCHIT dataset was omitted, as it differs from the others in that it does not link high-level artifacts with low-level artifacts.
An overview of the sets used can be seen in \autoref{tab:dataset_overview}.
\directQuote{Datasets comprise either high-level requirements (HLR), low-level requirements (LLR), requirements (R) or regulatory codes (RC). Percentage of linked source and target artifacts in the gold standard is given in brackets.}{hey2025RequirementsTraceability}

\begin{table}[]
    \centering
    \begin{tabular}{lccccc}
        & \multicolumn{2}{c}{Artifact Type} & \multicolumn{3}{c}{Number of Artifacts} \\
        \cmidrule(lr){ 2-3 } \cmidrule(lr){ 4-6 }
        Dataset   & Source & Target & Source     & Target     & \TLs \\
        \arrayrulecolor{kit-gray30} \midrule \arrayrulecolor{black}
        CM1-NASA  & HLR    & LLR    & 22 (86\%)  & 53 (57\%)  & 45   \\
        Dronology & HLR    & LLR    & 99 (93\%)  & 211 (99\%) & 220  \\
        GANNT     & HLR    & LLR    & 17 (100\%) & 69 (99\%)  & 68   \\
        MODIS     & HLR    & LLR    & 19 (63\%)  & 49 (63\%)  & 41   \\
        WARC      & HLR    & LLR    & 63 (95\%)  & 89 (89\%)  & 136  \\
        \arrayrulecolor{kit-gray30} \midrule \arrayrulecolor{black}
    \end{tabular}
    \caption{Overview of the datasets adjusted from \citeauthor{hey2025RequirementsTraceability}~\cite[Table 1]{hey2025RequirementsTraceability}}
    \label{tab:dataset_overview}
\end{table}


Several \LLMs will be used.
The focus will be on \OAI's \gpt and \gptmini models.
Compared to the locally hosted \codellama and \llama models by Meta AI, they enable faster evaluation with parallel requests instead of limitations through the host-hardware.
\API access is already implemented in the \LiSSAf.
In addition to the four models used by \citewithauthor{hey2025RequirementsTraceability} in their preceding work, current models will also be considered for evaluation.
\OAI recently introduced their \gptf model.

The \LiSSAf enables the usage of different embedding models for evaluation.
However, the singular embedding model which will be used is \ac{text-embedding} by \OAI.
As of this publication, it is the most up-to-date text embedding model by \OAI and was also used by \citeauthor{hey2025RequirementsTraceability}, thus improving comparability.

The similarity retriever in the \LiSSA pipeline will also not be modified for this evaluation.
The default cosine-similarity-retriever will be utilized.
\Todo{What is a cosine similarity retriever?}

Last but not least, the different \APE algorithms presented in \autoref{ch:Approach} will be added as an additional degree of freedom compared to the baseline evaluation.
Depending on the implementation, different variables, such as mainly the optimization prompt, will be introduced.
Unless specified, each evaluation will assume that the model used to optimize the prompt will be used to classify the \TLs later to derive the quantitative metrics for evaluation.


\section{Naive Prompt Optimization}
\label{sec:Evaluation:naive_optimization}

The initial thought people may have when thinking about prompt optimization is to ask the \LLM to optimize the prompt before usage.
To utilize this very simple approach, just two prerequisites are required.
Firstly, we need our prompt, which is to be optimized.
\autoref{prompt:yes_no} was chosen here as the initial prompt.
It is a \KISS binary classification prompt.
As the \LiSSAf has already employed this prompt for their simple classifier, I decided to start the optimization process with this prompt as well.
Secondly, an optimization prompt is needed.
Therefore, \autoref{prompt:initial_optimization} has been arbitrarily selected.
Unlike \autoref{prompt:yes_no}, this prompt has not been used in research yet to my knowledge.
The optimization prompt was designed by me, while implementing the naive prompt optimization approach.
Sentence completion through GitHub Copilot based on the \ac{gpt-copilot} model was also utilized.

\begin{prompt}{\KISS Original}{yes_no}
    \\
    Question: Here are two parts of software development artifacts.\\
\\
\{source\_type\}: \tripplequote \{source\_content\}\tripplequote\\
\\
\{target\_type\}: \tripplequote\{target\_content\}\tripplequote\\
Are they related?\\
\\
Answer with \textquotesingle{}yes\textquotesingle{} or  \textquotesingle{}no\textquotesingle{}.
\end{prompt}

\begin{prompt}{Simple Optimization Prompt}{initial_optimization}
    \\
    Optimize the following prompt to achieve better classification results for traceability link recovery.
Traceability links are to be found in the domain of \{source\_type\} to \{target\_type\}.
Do not modify the input and output formats specified by the original prompt.

Enclose your optimized prompt with <prompt></prompt> brackets.\\

The original prompt is provided below:\\
\tripplequote\{original\_prompt\}\tripplequote
\end{prompt}

With these two prompts the naive prompt optimization can be evaluated.
The results seemed to be quite consistent across different tested \LLMs.
I will present the optimized prompts by \OAI's \gpt model as an example of these.
\Todo{Add other models' outputs to the appendix?}

\subsection{Optimized Prompt Analysis}
\label{subsec:Evaluation:naive_optimization:optimized-prompt-analysis}
The first application of the optimization prompt usually yielded an inclusion of the source and target types into the text of the prompt.
This was possible as a set of training data was also provided for following, more sophisticated, prompt optimization approaches.
Further, the very simple classification question \Quote[\autoref{prompt:yes_no}]{Are they related?} was also expanded to be more specific for the \TLR problem.
The result can be seen in \autoref{prompt:yes_no_simple}.
The optimized prompt only depends on the input prompt and the domain of the source and target elements.
This very prompt thus is used for all simple \gpt evaluations in \autoref{tab:naive_optimization} for the domain of \RtR.

\begin{prompt}{\KISS Single Optimization Step}{yes_no_simple}
    \\
    \input{prompts/SimpleAndRepeatedOptimization/results-prompt-optimizationWARC_simple_gpt_gpt-4o-2024-08-06_0.json_411c69c3-cec3-3948-ad97-bb474536d21c}
\end{prompt}

As seen in \autoref{tab:naive_optimization}, application of the optimization prompt in the column \Quote[\autoref{tab:naive_optimization}]{simple} seems to generally improve the \fone.
By repeatedly applying the optimization prompt to the optimized prompt of the previous iteration, we can push this optimization further.
This yields \autoref{prompt:yes_no_iterative}
This particular prompt was again generated by \gpt using five iteration steps.
Once more, the behavior of different \LLMs is quite similar here.
We can see that mostly longer and more detailed instructions are included on how to find \TLs.

\begin{prompt}{\KISS Iterative Dumb Optimization}{yes_no_iterative}
    \\
    Question: You are given two parts of software development artifacts that need to be analyzed for traceability link recovery in the domain of requirement to requirement. Carefully examine the content provided below:\newline
\newline
\{source\_type\}: \textquotesingle{}\textquotesingle{}\textquotesingle{}\{source\_content\}\textquotesingle{}\textquotesingle{}\textquotesingle{}\newline
\newline
\{target\_type\}: \textquotesingle{}\textquotesingle{}\textquotesingle{}\{target\_content\}\textquotesingle{}\textquotesingle{}\textquotesingle{}\newline
\newline
Based on the information and context provided, determine if there is a traceability link between these two artifacts. Consider factors such as shared terminology, similar objectives, dependencies, and any direct or indirect relationships. Pay attention to the context and nuances in the language used, as well as any implicit connections that may not be immediately obvious. Evaluate the alignment of goals, the consistency of terminology, and the presence of any logical or functional dependencies. Are they related?\newline
\newline
Answer with \textquotesingle{}yes\textquotesingle{} or \textquotesingle{}no\textquotesingle{}.
\end{prompt}

However, unfortunately, this prompt does not necessarily perform better for our task.
The performance of this prompt, as seen in the \Quote[\autoref{tab:naive_optimization}]{iterative} column, seems to be generally worse than just the singular optimization application.
What is especially noteworthy, is that for \llama the classification fails completely with this prompt.
We can see in \autoref{prompt:yes_no_llama} that the prompt manged to modify the output format into different types of relationships.
The current classifiers of the \LiSSA project expect the \LLM output to include the classification result clearly as \textquotesingle yes \textquotesingle or \textquotesingle no \textquotesingle.
Thus, while this more detailed classification of the \LLM might even improve performance, it can not be properly evaluated.
\Todo{Consider how this might be mitiaged}

\begin{prompt}{\KISS Iterative Dumb Optimization \llama}{yes_no_llama}
    \\
    Question: Determine the direct or indirect link between two software development artifacts from the domain of requirements, focusing on connections within the scope of requirement\-to\-requirement traceability.\newline
\newline
            \{source\_type\}: \textquotesingle{}\textquotesingle{}\textquotesingle{}\{source\_content\}\textquotesingle{}\textquotesingle{}\textquotesingle{}\newline
\newline
            \{target\_type\}: \textquotesingle{}\textquotesingle{}\textquotesingle{}\{target\_content\}\textquotesingle{}\textquotesingle{}\textquotesingle{}\newline
\newline
            Classify the relationship as:\newline
            \- Direct: a clear and explicit connection between the two artifacts\newline
            \- Indirect via functional requirement: an indirect connection through one or more functional requirements\newline
            \- Indirect via non\-functional requirement: an indirect connection through one or more non\-functional requirements\newline
            \- No connection: no direct or indirect link exists between the two artifacts
\end{prompt}

\subsection{Systematic Evaluation Results}
\label{subsec:Evaluation:naive_optimization:systematic-evaluation-results}

The systematic evaluation results for varying models and datasets can be found in \autoref{tab:naive_optimization}.
The average and weighted average are also included.
The average value is computed across all datasets, while the weighted average factors the dataset size in as well.
The amount of \TLs in the gold standard reference for the dataset, seen in \autoref{tab:dataset_overview}, is used as the metric for dataset size.

\begin{table}
    \centering
    \renewcommand{\arraystretch}{1.4}
    \begin{tabular}{llcccccccccc}
\toprule
 & Prompt Group & \multicolumn{4}{c}{Original} & \multicolumn{3}{c}{Simple} & \multicolumn{3}{c}{Iterative} \\
 & Classifier Model & \makecell{GPT-4o} & \makecell{GPT-4o \\ mini} & \makecell{Code- \\ llama} & \makecell{Llama \\ 3.1} & \makecell{GPT-4o} & \makecell{GPT-4o \\ mini} & \makecell{Llama \\ 3.1} & \makecell{GPT-4o} & \makecell{GPT-4o \\ mini} & \makecell{Llama \\ 3.1} \\
Dataset & Metric &  &  &  &  &  &  &  &  &  &  \\
\midrule
\multirow[t]{3}{*}{\rotatebox{90}{CCHIT}} & F1 & 0.185 & 0.178 & 0.180 & 0.178 & 0.159 & 0.175 & 0.175 & 0.205 & 0.188 & 0.000 \\
 & Precision & 0.230 & 0.209 & 0.218 & 0.206 & 0.359 & 0.391 & 0.198 & 0.326 & 0.234 & 1.000 \\
 & Recall & 0.155 & 0.155 & 0.153 & 0.157 & 0.102 & 0.112 & 0.157 & 0.150 & 0.157 & 0.000 \\
\cline{1-12}
\multirow[t]{3}{*}{\rotatebox{90}{CM1-NASA}} & F1 & 0.453 & 0.443 & 0.446 & 0.436 & 0.562 & 0.545 & 0.436 & 0.487 & 0.443 & 0.000 \\
 & Precision & 0.349 & 0.337 & 0.341 & 0.330 & 0.529 & 0.656 & 0.330 & 0.392 & 0.337 & 1.000 \\
 & Recall & 0.644 & 0.644 & 0.644 & 0.644 & 0.600 & 0.467 & 0.644 & 0.644 & 0.644 & 0.000 \\
\cline{1-12}
\multirow[t]{3}{*}{\rotatebox{90}{GANNT}} & F1 & 0.548 & 0.544 & 0.526 & 0.552 & 0.581 & 0.544 & 0.544 & 0.565 & 0.548 & 0.000 \\
 & Precision & 0.552 & 0.544 & 0.538 & 0.561 & 0.643 & 0.596 & 0.544 & 0.587 & 0.552 & 1.000 \\
 & Recall & 0.544 & 0.544 & 0.515 & 0.544 & 0.529 & 0.500 & 0.544 & 0.544 & 0.544 & 0.000 \\
\cline{1-12}
\multirow[t]{3}{*}{\rotatebox{90}{ModisDataset}} & F1 & 0.237 & 0.237 & 0.161 & 0.244 & 0.321 & 0.170 & 0.188 & 0.386 & 0.324 & 0.000 \\
 & Precision & 0.212 & 0.212 & 0.238 & 0.224 & 0.600 & 0.667 & 0.145 & 0.688 & 0.407 & 1.000 \\
 & Recall & 0.268 & 0.268 & 0.122 & 0.268 & 0.220 & 0.098 & 0.268 & 0.268 & 0.268 & 0.000 \\
\cline{1-12}
\multirow[t]{3}{*}{\rotatebox{90}{WARC}} & F1 & 0.504 & 0.491 & 0.491 & 0.486 & 0.598 & 0.590 & 0.490 & 0.561 & 0.497 & 0.000 \\
 & Precision & 0.394 & 0.378 & 0.429 & 0.377 & 0.545 & 0.520 & 0.377 & 0.479 & 0.388 & 1.000 \\
 & Recall & 0.699 & 0.699 & 0.574 & 0.684 & 0.662 & 0.684 & 0.699 & 0.676 & 0.691 & 0.000 \\
\cline{1-12}
\multirow[t]{3}{*}{\rotatebox{90}{dronology}} & F1 & 0.504 & 0.498 & 0.495 & 0.489 & 0.587 & 0.506 & 0.497 & 0.526 & 0.500 & 0.000 \\
 & Precision & 0.395 & 0.387 & 0.413 & 0.382 & 0.529 & 0.430 & 0.386 & 0.427 & 0.392 & 1.000 \\
 & Recall & 0.695 & 0.695 & 0.618 & 0.682 & 0.659 & 0.614 & 0.695 & 0.686 & 0.691 & 0.000 \\
\cline{1-12}
\bottomrule
\end{tabular}

    \renewcommand{\arraystretch}{1}
    \caption{Naive prompt optimization approach prompting the model to optimize the classification prompt}
    \label{tab:naive_optimization}
\end{table}

\newpage


\section{Simple Feedback Optimization}
\label{sec:Evaluation:simple_feedback_optimization}

The following parameters will be used to describe prompts for this section.
They can be used as configuration parameters for the feedback optimization process.

\begin{align*}
    mi\coloneqq  & \text{maximum iterations}\\
    & \text{the prompt will be optimized $mi$ times} \\
    n \coloneqq & \text{feedback sample size}\\
    & \text{$n$ examples of misclassified \TLs will be provided}
\end{align*}

\begin{prompt}{Feedback Prompt}{feedback_initial}
    \\
    The current prompt is not performing well in classifying the following trace links.
To help you improve the prompt, I will provide examples of misclassified trace links. 
Please analyze these examples and adjust the prompt accordingly. 
The examples are as follows: \\
\{identifier\}\\
\{source\_type\}: \tripplequote\{source\_content\}\tripplequote\\
\{target\_type\}: \tripplequote\{target\_content\}\tripplequote\\
Classification result: \{classification\}
\end{prompt}

\subsection{Optimized Prompt Analysis}
\label{subsec:Evaluation:simple_feedback:optimized-prompt-analysis}

\begin{prompt}{Optimized Prompt n = 3, mi = 1}{feedback_optimized_3_1}
    \\
    Question: Here are two parts of software development artifacts.\newline
\newline
\{source\_type\}: \textquotesingle{}\textquotesingle{}\textquotesingle{}\{source\_content\}\textquotesingle{}\textquotesingle{}\textquotesingle{}\newline
\newline
\{target\_type\}: \textquotesingle{}\textquotesingle{}\textquotesingle{}\{target\_content\}\textquotesingle{}\textquotesingle{}\textquotesingle{}\newline
\newline
Consider the following when determining if they are related:\newline
1. Look for shared terminology or concepts, such as specific functions, interfaces, or standards mentioned in both requirements.\newline
2. Assess whether one requirement describes a feature or functionality that directly supports or complements the other.\newline
3. Determine if there is a dependency or a logical connection between the two requirements, such as one requirement ensuring compatibility or compliance that the other relies on.\newline
\newline
Are they related?\newline
\newline
Answer with \textquotesingle{}yes\textquotesingle{} or \textquotesingle{}no\textquotesingle{}.
\end{prompt}

\begin{prompt}{Optimized Prompt n = 5, mi = 10}{feedback_optimized_5_10}
    \\
    Question: Here are two parts of software development artifacts.\newline
\newline
\{source\_type\}: \textquotesingle{}\textquotesingle{}\textquotesingle{}\{source\_content\}\textquotesingle{}\textquotesingle{}\textquotesingle{}\newline
\newline
\{target\_type\}: \textquotesingle{}\textquotesingle{}\textquotesingle{}\{target\_content\}\textquotesingle{}\textquotesingle{}\textquotesingle{}\newline
\newline
Consider the following when determining if they are related:\newline
1. Identify shared terminology or concepts, such as specific functions, interfaces, standards, or types of records mentioned in both requirements. Pay attention to synonyms or closely related terms that may indicate a connection.\newline
2. Evaluate whether one requirement describes a functionality or feature that directly supports, complements, or is necessary for the implementation of the other. Consider if one requirement provides a foundation or prerequisite for the other.\newline
3. Determine if there is a dependency, logical connection, or encapsulation relationship between the two requirements, such as one requirement ensuring compatibility, compliance, or abstraction that the other relies on. Consider if changes in one requirement would impact the other.\newline
4. Assess if both requirements contribute to a common goal, objective, or aspect of the software system, such as ensuring standard compliance, providing a universal interface, or enhancing system robustness.\newline
5. Consider if the requirements address different aspects of the same feature or functionality, such as creation, management, or processing of records. Look for complementary roles or responsibilities that together fulfill a broader requirement.\newline
6. Analyze if the requirements involve similar actions or processes, such as creating, managing, or processing records, and if they are part of a sequence or workflow that connects them.\newline
7. Reflect on whether the requirements share a common context or domain, which might imply a relationship even if the direct connection is not immediately obvious.\newline
8. Pay special attention to the presence of universal interfaces, encapsulation, and compliance with standards, as these are critical aspects that often indicate a relationship between requirements in this domain.\newline
9. Examine if both requirements mention or imply the use of a universal header or interface, as this is a key indicator of a relationship in this context.\newline
10. Consider if the requirements involve the same or related types of WARC records, as this can suggest a direct or indirect relationship.\newline
11. Look for explicit mentions of memory management, remote management, or other specific functionalities that might indicate a relationship, even if not directly related to WARC records.\newline
12. Consider the broader context of the requirements, such as the overall architecture or design principles they support, to identify indirect relationships.\newline
13. Pay attention to the specific phrasing and structure of the requirements, as similar language or structure can indicate a relationship.\newline
14. Consider the intent and purpose behind each requirement, as aligning intents can suggest a relationship even if the specifics differ.\newline
15. Evaluate if the requirements mention or imply the need for compatibility or interoperability, as this can indicate a relationship.\newline
16. Consider if the requirements address scalability, performance, or optimization aspects that might be interrelated.\newline
17. Look for any implicit or explicit dependencies that might not be immediately obvious but are crucial for the system\textquotesingle{}s functionality.\newline
18. Pay attention to the specific types of WARC records mentioned and whether they are addressed in both requirements, as this can indicate a direct relationship.\newline
19. Consider if the requirements mention or imply the need for a single header file or universal interface, as this is a critical aspect in this domain.\newline
20. Reflect on whether the requirements address encapsulation or abstraction, as these are key concepts that often indicate a relationship.\newline
\newline
Are they related?\newline
\newline
Answer with \textquotesingle{}yes\textquotesingle{} or \textquotesingle{}no\textquotesingle{}.
\end{prompt}

\subsection{Systematic Evaluation Results}
\label{subsec:Evaluation:simple_feedback_optimization:systematic-evaluation-results}

\begin{landscape}
    \begin{table}
        \centering
        %TODO: Careful, currently the tabularx width is manually replaced with \hsize in the table to adjust to landscape mode
        \renewcommand{\arraystretch}{1}
        %! Author = ???
%! Date = 2025-08-13

\begin{tabularx}{\textwidth}{p{1.3em}l Z  Z  Z  Z  Z  Z  Z  Z  Z  Z  Z  Z  Z  Z  Z }
    \toprule
    \multirow{2}{1em}[-0.45em]{\rotatebox{90}{Dataset}} & \multirow{2}{*}[-0.9em]{\rotatebox{90}{Metric}}  & \multicolumn{ 3 }{c}{KISS-Original}  & \multicolumn{ 3 }{c}{feedback (mi=1, n=0)}  & \multicolumn{ 3 }{c}{feedback (mi=10, n=5)}  & \multicolumn{ 3 }{c}{feedback (mi=5, n=3)}  & \multicolumn{ 3 }{c}{unknown}                                                                                          \\
        \cmidrule(lr){ 3-5 }
        \cmidrule(lr){ 6-8 }
        \cmidrule(lr){ 9-11 }
        \cmidrule(lr){ 12-14 }
        \cmidrule(lr){ 15-17 }
    &                            & Codellama        & GPT-4o-mini        & Llama 3.1        & Codellama        & GPT-4o-mini        & Llama 3.1        & Codellama        & GPT-4o-mini        & Llama 3.1        & Codellama        & GPT-4o-mini        & Llama 3.1        & Codellama        & GPT-4o-mini        & Llama 3.1           \\
    \midrule
        \multirow{3}{*}{\rotatebox{90}{CCHIT}}
                & P.                     & 0.242  & 0.0  & 0.206  & -  & 0.0  & -  & -  & 0.0  & -  & -  & 0.0  & -  & -  & 0.267  & -  \\
                & R.                     & 0.153     & 0.0     & 0.157     & -     & 0.0     & -     & -     & 0.0     & -     & -     & 0.0     & -     & -     & 0.155     & -     \\
                & F\textsubscript{1}     & 0.188         & 0.0         & 0.178         & -         & 0.0         & -         & -         & 0.0         & -         & -         & 0.0         & -         & -         & 0.196         & -         \\
                \arrayrulecolor{kit-gray30}     \midrule     \arrayrulecolor{black}
        \multirow{3}{*}{\rotatebox{90}{CM1-NASA}}
                & P.                     & 0.333  & 0.0  & -  & -  & 0.0  & -  & -  & 0.0  & -  & -  & 0.0  & -  & -  & 0.575  & -  \\
                & R.                     & 0.644     & 0.0     & -     & -     & 0.0     & -     & -     & 0.0     & -     & -     & 0.0     & -     & -     & 0.511     & -     \\
                & F\textsubscript{1}     & 0.439         & 0.0         & -         & -         & 0.0         & -         & -         & 0.0         & -         & -         & 0.0         & -         & -         & 0.541         & -         \\
                \arrayrulecolor{kit-gray30}     \midrule     \arrayrulecolor{black}
        \multirow{3}{*}{\rotatebox{90}{GANNT}}
                & P.                     & 0.544  & 0.544  & -  & -  & 0.544  & -  & -  & 0.389  & -  & -  & 0.561  & -  & -  & 0.544  & -  \\
                & R.                     & 0.544     & 0.544     & -     & -     & 0.544     & -     & -     & 0.103     & -     & -     & 0.544     & -     & -     & 0.544     & -     \\
                & F\textsubscript{1}     & 0.544         & 0.544         & -         & -         & 0.544         & -         & -         & 0.163         & -         & -         & 0.552         & -         & -         & 0.544         & -         \\
                \arrayrulecolor{kit-gray30}     \midrule     \arrayrulecolor{black}
        \multirow{3}{*}{\rotatebox{90}{ModisDataset}}
                & P.                     & 0.275  & 0.0  & -  & -  & 0.0  & -  & -  & 0.0  & -  & -  & 0.0  & -  & -  & 0.444  & -  \\
                & R.                     & 0.268     & 0.0     & -     & -     & 0.0     & -     & -     & 0.0     & -     & -     & 0.0     & -     & -     & 0.098     & -     \\
                & F\textsubscript{1}     & 0.272         & 0.0         & -         & -         & 0.0         & -         & -         & 0.0         & -         & -         & 0.0         & -         & -         & 0.16         & -         \\
                \arrayrulecolor{kit-gray30}     \midrule     \arrayrulecolor{black}
        \multirow{3}{*}{\rotatebox{90}{WARC}}
                & P.                     & 0.442  & 0.38  & -  & -  & 0.377  & -  & -  & 0.377  & -  & -  & 0.382  & -  & -  & 0.0  & -  \\
                & R.                     & 0.309     & 0.699     & -     & -     & 0.699     & -     & -     & 0.699     & -     & -     & 0.691     & -     & -     & 0.0     & -     \\
                & F\textsubscript{1}     & 0.364         & 0.492         & -         & -         & 0.49         & -         & -         & 0.49         & -         & -         & 0.492         & -         & -         & 0.0         & -         \\
                \arrayrulecolor{kit-gray30}     \midrule     \arrayrulecolor{black}
        \multirow{3}{*}{\rotatebox{90}{dronology}}
                & P.                     & 0.379  & 0.387  & -  & -  & 0.386  & -  & -  & 0.39  & -  & -  & 0.388  & -  & -  & 0.387  & -  \\
                & R.                     & 0.591     & 0.695     & -     & -     & 0.695     & -     & -     & 0.691     & -     & -     & 0.691     & -     & -     & 0.695     & -     \\
                & F\textsubscript{1}     & 0.462         & 0.498         & -         & -         & 0.497         & -         & -         & 0.498         & -         & -         & 0.497         & -         & -         & 0.498         & -         \\
                \arrayrulecolor{kit-gray30}     \midrule     \arrayrulecolor{black}
        \multirow{3}{*}{\rotatebox{90}{Avg.}}
                & P.                     & 0.369  & 0.218  & 0.206  & -  & 0.218  & -  & -  & 0.193  & -  & -  & 0.222  & -  & -  & 0.369  & -  \\
                & R.                     & 0.418     & 0.323     & 0.157     & -     & 0.323     & -     & -     & 0.249     & -     & -     & 0.321     & -     & -     & 0.334     & -     \\
                & F\textsubscript{1}     & 0.378         & 0.256         & 0.178         & -         & 0.255         & -         & -         & 0.192         & -         & -         & 0.257         & -         & -         & 0.323         & -         \\
                \arrayrulecolor{kit-gray30}     \midrule     \arrayrulecolor{black}
\end{tabularx}
        \renewcommand{\arraystretch}{1}
        \caption{Naive prompt optimization approach considering previous misclassified \TLs}
        \label{tab:placeholder}
    \end{table}
\end{landscape}


\section{Varying the Optimization Prompt}
\label{sec:Evaluation:varying-the-optimization-prompt}
In a recent paper by \citewithauthor{zadenoori2025AutomaticPrompt} the authors \Todo{Text needs to be added}.
Their prompt is designed to optimize classification prompts by enhancing the explanations of categories within the prompt.
As it also utilizes feedback from misclassified \TLs, it can be used as an alternative optimization prompt for the naive feedback optimization approach.

They used the following optimization prompt \autoref{prompt:zadenoori_optimization}:
\begin{prompt}{Zadenoori Optimization Prompt}{zadenoori_optimization}
    \\
    \begin{prompt}{Zadenoori Optimization Prompt}{zadenoori_optimization}\\
     You are required to enhance and clarify the explanations of the categories in the prompt by integrating illustrative examples and information implicitly referenced in the initial context. \\
     The optimized prompt must follow these strict guidelines: \\
     Maintain the Original Steps: The steps in the optimized prompt must remain exactly the same as in the sample prompt; no changes should be made to the steps’ structure or order. \\
     Expand Explanations: Enrich and expand the explanations of each category within the steps, incorporating examples provided.
     Use these examples to enhance understanding and provide clarity, but ensure all content remains within the existing steps and does not extend beyond them. \\
     Incorporate Class Explanations: Specifically, integrate the detailed "Class Explanations" of categories from the first prompt into the optimized prompt.
     For each category, introduce implicit clarifications based on relevant data extracted from the context, keeping all additions within the boundaries of the original steps. \\
     End Strictly After Step 5: The optimized prompt must strictly end after step 5.
     Do not add any additional steps, conclusions, or content beyond this point. \\
\end{prompt}
\end{prompt}

As their optimization prompt is more detailed than the one used in \autoref{sec:Evaluation:simple_feedback_optimization}, it is used as an alternative optimization prompt for the naive optimization approach.
As seen in \autoref{tab:zadenoori_optimization} the results of this optimization approach are quite similar to the ones of \autoref{sec:Evaluation:naive_optimization}.

\newpage
\begin{prompt}{Opimized WARC prompt}{warc_zandoori_optimized}
    \\
    Question: Here are two parts of software development artifacts.\newline
\newline
\{source\_type\}: \textquotesingle{}\textquotesingle{}\textquotesingle{}\{source\_content\}\textquotesingle{}\textquotesingle{}\textquotesingle{}\newline
\newline
\{target\_type\}: \textquotesingle{}\textquotesingle{}\textquotesingle{}\{target\_content\}\textquotesingle{}\textquotesingle{}\textquotesingle{}\newline
Are they related?\newline
\newline
Answer with \textquotesingle{}yes\textquotesingle{} or \textquotesingle{}no\textquotesingle{}.\newline
\newline
Class Explanations:\newline
\newline
1. \*\*Yes\*\*: This classification is used when the two artifacts are directly related or connected in a meaningful way. For example, if one requirement specifies a feature or functionality that is directly supported or enabled by another requirement, they should be classified as \textquotesingle{}Yes\textquotesingle{}. In the provided examples, if a requirement specifies the need for a universal interface to create WARC records and another requirement details the types of WARC records that can be created through such an interface, they are related and should be classified as \textquotesingle{}Yes\textquotesingle{}. This implies a direct relationship where one requirement\textquotesingle{}s implementation or purpose is explicitly supported or fulfilled by the other. For instance, if "FR 3" specifies providing functions through a universal interface for creating WARC records, and "SRS 7" specifies the types of WARC records that can be created through such an interface, they are directly related. This direct relationship is characterized by one requirement enabling or detailing the implementation of another, ensuring that the functionality or purpose is clearly aligned and supported.\newline
\newline
2. \*\*No\*\*: This classification is used when the two artifacts are not directly related or do not have a meaningful connection. For instance, if one requirement discusses the encapsulation of internal functionality and another requirement focuses on ensuring compatibility between versions, they address different aspects and should be classified as \textquotesingle{}No\textquotesingle{}. In the examples provided, if one requirement specifies the need for a single header file and another requirement details the types of WARC records, they are not directly related and should be classified as \textquotesingle{}No\textquotesingle{}. This indicates that the requirements do not share a direct or explicit connection in terms of functionality or purpose, even if they are part of the same broader system or project. For example, "FR 1" requiring a single header file and "SRS 7" detailing WARC record types do not directly relate, as they address different aspects of the system. The lack of a direct relationship is evident when the requirements focus on separate functionalities or objectives, without one directly supporting or fulfilling the other.\newline
\newline
Incorporate these explanations to ensure accurate classification by identifying direct relationships or lack thereof between the requirements.
\end{prompt}

As their optimization prompt is more detailed than the one used in \autoref{sec:Evaluation:simple_feedback_optimization} , it is used as an alternative optimization prompt for the naive optimization approach.
As seen in \autoref{tab:zadenoori_optimization} the results of this optimization approach are quite similar to the ones of \autoref{sec:Evaluation:naive_optimization}.

\begin{table}
    \centering
    \renewcommand{\arraystretch}{1.4}
    
\begin{tabularx}{\textwidth}{p{1.3em}l Z  Z  Z  Z  Z  Z  Z  Z }
    \toprule
    \multirow{2}{1em}[-0.45em]{\rotatebox{90}{Dataset}} & \multirow{2}{*}[-0.9em]{\rotatebox{90}{Metric}}  & \multicolumn{ 2 }{c}{KISS-Original}  & \multicolumn{ 2 }{c}{feedback (iter=1, n=3)}  & \multicolumn{ 2 }{c}{feedback (iter=3, n=1)}  & \multicolumn{ 2 }{c}{feedback (iter=5, n=3)}                                                                                          \\
    \cmidrule(lr){ 3-4 }
    \cmidrule(lr){ 5-6 }
    \cmidrule(lr){ 7-8 }
    \cmidrule(lr){ 9-10 }
    &                            & GPT-4o        & GPT-4o-mini        & GPT-4o        & GPT-4o-mini        & GPT-4o        & GPT-4o-mini        & GPT-4o        & GPT-4o-mini           \\
    \midrule
    \multirow{ 3 }{*}{\rotatebox{90}{CCHIT}}
    & P.    & 0.23    & 0.209    & \textbf{ 0.311 }    & 0.262    & 0.304    & 0.225    & 0.294    & 0.251 \\
    & R.    & \textbf{ 0.155 }    & \textbf{ 0.155 }    & 0.148    & 0.153    & 0.148    & \textbf{ 0.155 }    & 0.148    & 0.153 \\
    & F\textsubscript{1}    & 0.185    & 0.178    & \textbf{ 0.201 }    & 0.193    & 0.199    & 0.184    & 0.197    & 0.19 \\
    \arrayrulecolor{kit-gray30} \midrule \arrayrulecolor{black}

    \multirow{ 3 }{*}{\rotatebox{90}{CM1-NASA}}
    & P.    & 0.349    & 0.337    & 0.354    & \textbf{ 0.372 }    & 0.362    & 0.358    & 0.354    & 0.358 \\
    & R.    & \textbf{ 0.644 }    & \textbf{ 0.644 }    & \textbf{ 0.644 }    & \textbf{ 0.644 }    & \textbf{ 0.644 }    & \textbf{ 0.644 }    & \textbf{ 0.644 }    & \textbf{ 0.644 } \\
    & F\textsubscript{1}    & 0.453    & 0.443    & 0.457    & \textbf{ 0.472 }    & 0.464    & 0.46    & 0.457    & 0.46 \\
    \arrayrulecolor{kit-gray30} \midrule \arrayrulecolor{black}

    \multirow{ 3 }{*}{\rotatebox{90}{GANNT}}
    & P.    & 0.552    & 0.544    & 0.552    & 0.544    & \textbf{ 0.569 }    & 0.544    & 0.552    & \textbf{ 0.569 } \\
    & R.    & \textbf{ 0.544 }    & \textbf{ 0.544 }    & \textbf{ 0.544 }    & \textbf{ 0.544 }    & \textbf{ 0.544 }    & \textbf{ 0.544 }    & \textbf{ 0.544 }    & \textbf{ 0.544 } \\
    & F\textsubscript{1}    & 0.548    & 0.544    & 0.548    & 0.544    & \textbf{ 0.556 }    & 0.544    & 0.548    & \textbf{ 0.556 } \\
    \arrayrulecolor{kit-gray30} \midrule \arrayrulecolor{black}

    \multirow{ 3 }{*}{\rotatebox{90}{ModisDataset}}
    & P.    & 0.212    & 0.212    & \textbf{ 0.562 }    & 0.324    & 0.314    & 0.5    & 0.444    & 0.275 \\
    & R.    & \textbf{ 0.268 }    & \textbf{ 0.268 }    & 0.22    & \textbf{ 0.268 }    & \textbf{ 0.268 }    & \textbf{ 0.268 }    & 0.195    & \textbf{ 0.268 } \\
    & F\textsubscript{1}    & 0.237    & 0.237    & 0.316    & 0.293    & 0.289    & \textbf{ 0.349 }    & 0.271    & 0.272 \\
    \arrayrulecolor{kit-gray30} \midrule \arrayrulecolor{black}

    \multirow{ 3 }{*}{\rotatebox{90}{WARC}}
    & P.    & 0.394    & 0.38    & 0.455    & 0.378    & 0.44    & 0.392    & \textbf{ 0.485 }    & 0.383 \\
    & R.    & \textbf{ 0.699 }    & \textbf{ 0.699 }    & \textbf{ 0.699 }    & 0.691    & \textbf{ 0.699 }    & 0.669    & \textbf{ 0.699 }    & 0.684 \\
    & F\textsubscript{1}    & 0.504    & 0.492    & 0.551    & 0.488    & 0.54    & 0.495    & \textbf{ 0.572 }    & 0.491 \\
    \arrayrulecolor{kit-gray30} \midrule \arrayrulecolor{black}

    \multirow{ 3 }{*}{\rotatebox{90}{dronology}}
    & P.    & 0.395    & 0.387    & 0.459    & 0.408    & 0.422    & 0.408    & 0.436    & \textbf{ 0.475 } \\
    & R.    & \textbf{ 0.695 }    & \textbf{ 0.695 }    & 0.686    & 0.686    & 0.691    & 0.686    & 0.686    & 0.614 \\
    & F\textsubscript{1}    & 0.504    & 0.498    & \textbf{ 0.55 }    & 0.512    & 0.524    & 0.512    & 0.534    & 0.536 \\
    \arrayrulecolor{kit-gray30} \midrule \arrayrulecolor{black}

    \multirow{ 3 }{*}{\rotatebox{90}{Avg.}}
    & P.    & 0.355    & 0.345    & \textbf{ 0.449 }    & 0.381    & 0.402    & 0.405    & 0.427    & 0.385 \\
    & R.    & \textbf{ 0.501 }    & \textbf{ 0.501 }    & 0.49    & 0.498    & 0.499    & 0.494    & 0.486    & 0.485 \\
    & F\textsubscript{1}    & 0.405    & 0.399    & \textbf{ 0.437 }    & 0.417    & 0.429    & 0.424    & 0.43    & 0.417 \\
    \arrayrulecolor{kit-gray30} \midrule \arrayrulecolor{black}

    \multirow{ 3 }{*}{\rotatebox{90}{Weighted Avg.}}
    & P.    & 0.308    & 0.292    & \textbf{ 0.385 }    & 0.33    & 0.364    & 0.318    & 0.37    & 0.337 \\
    & R.    & \textbf{ 0.379 }    & \textbf{ 0.379 }    & 0.372    & 0.375    & 0.375    & 0.374    & 0.371    & 0.36 \\
    & F\textsubscript{1}    & 0.324    & 0.317    & \textbf{ 0.351 }    & 0.33    & 0.343    & 0.328    & 0.346    & 0.334 \\
    \arrayrulecolor{kit-gray30} \midrule \arrayrulecolor{black}

\end{tabularx}
    \renewcommand{\arraystretch}{1}
    \caption{Naive prompt optimization approach using the optimization prompt by \citewithauthor{zadenoori2025AutomaticPrompt}}
    \label{tab:zadenoori_optimization}
\end{table}