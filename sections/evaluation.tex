%% LaTeX2e class for student theses
%% sections/evaluation.tex
%% 
%% Karlsruhe Institute of Technology
%% Institute of Information Security and Dependability
%% Software Design and Quality (SDQ)
%%
%% Dr.-Ing. Erik Burger
%% burger@kit.edu
%%
%% Version 1.6, 2024-06-07

\chapter{Evaluation}
\label{ch:Evaluation}

\Todo{Talk about evaluation in general}

\section{Setup}
\label{sec:Evaluation:setup}
To evaluate the \APE algorithms proposed in \Todo{add link to approach}, a multitude of different datasets will be used.
They are taken from \citewithauthor{hey2025ReplicationPackage}'s replication package.
Some of the gold standard files were modified to provide consistency between artifact naming and the gold standard reference.
The actual contents are not affected by this.
The CCHIT dataset was omitted, as it differs from the others in that it does not link high-level artifacts with low-level artifacts.
An overview of the sets used can be seen in \autoref{tab:dataset_overview}.
\directQuote{Datasets comprise either high-level requirements (HLR), low-level requirements (LLR), requirements (R) or regulatory codes (RC). Percentage of linked source and target artifacts in the gold standard is given in brackets.}{hey2025RequirementsTraceability}

\begin{table}[]
    \centering
    \begin{tabular}{lccccc}
                 & \multicolumn{2}{c}{Artifact Type} & \multicolumn{3}{c}{Number of Artifacts} \\
        \cmidrule(lr){ 2-3 } \cmidrule(lr){ 4-6 }
         Dataset    & Source& Target& Source    & Target    & \TLs \\
        \arrayrulecolor{kit-gray30} \midrule \arrayrulecolor{black}
         CM1-NASA   & HLR   & LLR   & 22 (86\%) & 53 (57\%) & 45 \\
         Dronology  & HLR   & LLR   & 99 (93\%) & 211 (99\%)& 220 \\
         GANNT      & HLR   & LLR   & 17 (100\%)& 69 (99\%) & 68 \\
         MODIS      & HLR   & LLR   & 19 (63\%) & 49 (63\%) & 41 \\
         WARC       & HLR   & LLR   & 63 (95\%) & 89 (89\%) & 136 \\
        \arrayrulecolor{kit-gray30} \midrule \arrayrulecolor{black}
    \end{tabular}
    \caption{Overview of the datasets adjusted from \citeauthor{hey2025RequirementsTraceability} \cite[Table 1]{hey2025RequirementsTraceability}}
    \label{tab:dataset_overview}
\end{table}


Several \LLMs will be used.
The focus will be on \OAI's \gpt and \gptmini models.
Compared to the locally hosted \codellama and \llama models by Meta AI, they enable faster evaluation with parallel requests.
\API access is already implemented in the \LiSSAf.
In addition the the four models used by \citewithauthor{hey2025RequirementsTraceability} in their preceding work, current models will also be considered for evaluation.
\OAI recently introduced their \gptf model.

The \LiSSAf enables the usage of different embedding models for evaluation.
However, the singular embedding model which will be used is \ac{text-embedding} by \OAI.
As of this publication, it is the most up-to-date text embedding model by \OAI and was also used by \citeauthor{hey2025RequirementsTraceability}, thus improving comparability.

The similarity retriever in the \LiSSA pipeline will also not be modified for this evaluation.
The default cosine-similarity-retriever will be utilized.
\Todo{What is a cosine similarity retriever?}

Last but not least, the different \APE algorithms presented in \autoref{ch:Approach} will be added as an additional degree of freedom compared to the baseline evaluation.
Depending on the implementation, different variables, such as mainly the optimization prompt, will be introduced.
Unless specified, each evaluation will assume that the model used to optimize the prompt will be used to classify the \TLs later to derive the quantitative metrics for evaluation.


\section{Naive Prompt Optimization}
\label{sec:Evaluation:naive_optimization}

The initial thought people may have when thinking about prompt optimization is to ask the \LLM to optimize the prompt before usage. To utilize this very simple approach, just two prerequisites are required. Firstly, we need our prompt, which is to be optimized. \autoref{prompt:yes_no} was chosen here as the initial prompt. It is a \KISS binary classification prompt. As the \LiSSAf has already employed this prompt for their simple classifier, I decided to start the optimization process with this prompt as well. Secondly, an optimization prompt is needed. Herefore \autoref{prompt:initial_optimization} has been arbitrarily selected. Unlike \autoref{prompt:yes_no}, this prompt has not been used in reasearch yet to my knowledge.
The optimization prompt was designed by me while implementing the naive prompt optimization approach.

\begin{prompt}{\KISS Original}{yes_no}\\
Question: Here are two parts of software development artifacts.\\
\{source\_type\}: \tripplequote \{source\_content\}\tripplequote\\
\{target\_type\}: \tripplequote\{target\_content\}\tripplequote\\
Are they related?\\
Answer with \textquotesingle{}yes\textquotesingle{} or  \textquotesingle{}no\textquotesingle{}.
\end{prompt}

\begin{prompt}{Simple Optimization Prompt}{initial_optimization}\\
    Optimize the following prompt to achieve better classification results for traceability link recovery. \\
    Traceability links are to be found in the domain of \{source\_type\} to \{target\_type\}. \\
    Do not modify the input and output formats specified by the original prompt.\\
    Enclose your optimized prompt with <prompt></prompt> brackets.\\
    The original prompt is provided below:\\
    \tripplequote\{original\_prompt\}\tripplequote
\end{prompt}

With these two prompts the naive prompt optimization can be evaluated. The results seemed to be quite consistent across different tested \LLMs. I will present the optimized prompts by OpenAI's GPT-4o model as an example of these.
\Todo{Add other models' outputs to the appendix?}

\subsection{Optimized Prompt Analysis}
The first application of the optimization prompt usually yielded an inclusion of the source and target types into the text of the prompt. This was possible as a set of training data was also provided for later, more sophisticated prompt optimization approaches. Further, the very simple classification question "Are they related?" was also expanded to be more specific for the \TLR problem.  The result can be seen in \autoref{prompt:yes_no_simple}.
The optimized prompt only depends on the input prompt and the domain of the source and target elements. This very prompt thus is used for all simple GPT-4o evaluations in \autoref{tab:naive_optimization} for the domain of \RtR. 

\begin{prompt}{\KISS Single Optimization Step}{yes_no_simple}\\
    Question: Below are two components from software development artifacts that need to be analyzed for traceability.\\

            Source Requirement \{source\_type\}: \tripplequote\{source\_content\}\tripplequote

            Target Requirement \{target\_type\}: \tripplequote\{target\_content\}\tripplequote
            
            Based on the content and context, determine if there is a traceability link between them.

            Answer with \textquotesingle{}yes\textquotesingle{} or \textquotesingle{}no\textquotesingle{}.
    
\end{prompt}

As seen in \autoref{tab:naive_optimization}, application of the optimization prompt seems to generally improve the $f_1-score$. 
By repeatedly applying the optimization prompt to the optimized prompt of the previous iteration, we can push this optimization further.
This yields \autoref{prompt:yes_no_iterative}
Once more, the behavior of different \LLMs is quite similar here.
We can see that mostly longer and more detailed instructions are included on how to find \TLs. 
However, unfortunately, this prompt does not necessarily perform better for our task. 
The performance of this prompt, as seen in \autoref{tab:naive_optimization}, seems to be generally worse than just the singular optimization application.

\begin{prompt}{\KISS Single Iterative Dumb Optimization}{yes_no_iterative}\\
    Question: You are given two components from software development artifacts that need to be analyzed for traceability link recovery in the domain of requirement to requirement.\\
    
    \{source\_type\}: \tripplequote\{source\_content\}\tripplequote

    \{target\_type\}: \tripplequote\{target\_content\}\tripplequote

    Carefully analyze both components to determine if a traceability link exists between them. Focus on identifying shared terminology, dependencies, logical connections, and alignment of objectives and constraints. Look for direct references, similar phrasing, or common goals that might indicate a connection. Evaluate the consistency of requirements, the presence of complementary or supplementary information, and any historical or contextual relationships that could suggest a link. Consider the intent and purpose behind each requirement to ensure a thorough analysis. Pay special attention to any implicit connections that may not be immediately obvious but are crucial for understanding the relationship. Use a systematic approach to ensure no potential link is overlooked. Prioritize clarity and precision in your analysis to enhance the accuracy of your classification.

    Answer with \textquotesingle{}yes\textquotesingle{} or \textquotesingle{}no\textquotesingle{}.
    
\end{prompt}

\subsection{Systematic Evaluation Results}

\begin{table}
    \centering
    \renewcommand{\arraystretch}{1.4}
    %! Author = Daniel Schwab
%! Date = 2025-08-30

\begin{tabularx}{\textwidth}{p{1.3em}l Z  Z  Z  Z  Z  Z  Z  Z  Z }
    \toprule
    \multirow{2}{1em}[-0.45em]{\rotatebox{90}{Dataset}} & \multirow{2}{*}[-0.9em]{\rotatebox{90}{Metric}}  & \multicolumn{ 3 }{c}{KISS-Original}  & \multicolumn{ 3 }{c}{simple}  & \multicolumn{ 3 }{c}{iterative (iter=5)}                                                                                          \\
    \cmidrule(lr){ 3-5 }
    \cmidrule(lr){ 6-8 }
    \cmidrule(lr){ 9-11 }
    &                            & GPT-4o        & GPT-4o-mini        & Llama 3.1        & GPT-4o        & GPT-4o-mini        & Llama 3.1        & GPT-4o        & GPT-4o-mini        & Llama 3.1           \\
    \midrule
    \multirow{ 3 }{*}{\rotatebox{90}{CM1-NASA}}
    & P.    & 0.349    & 0.337    & 0.33    & 0.354    & 0.622    & 0.33    & 0.341    & 0.333    & \textbf{ 1.0 } \\
    & R.    & \textbf{ 0.644 }    & \textbf{ 0.644 }    & \textbf{ 0.644 }    & \textbf{ 0.644 }    & 0.511    & \textbf{ 0.644 }    & \textbf{ 0.644 }    & \textbf{ 0.644 }    & 0.0 \\
    & F\textsubscript{1}    & 0.453    & 0.443    & 0.436    & 0.457    & \textbf{ 0.561 }    & 0.436    & 0.446    & 0.439    & 0.0 \\
    \arrayrulecolor{kit-gray30} \midrule \arrayrulecolor{black}

    \multirow{ 3 }{*}{\rotatebox{90}{GANNT}}
    & P.    & 0.552    & 0.544    & 0.561    & 0.578    & 0.6    & 0.544    & 0.578    & 0.544    & \textbf{ 1.0 } \\
    & R.    & \textbf{ 0.544 }    & \textbf{ 0.544 }    & \textbf{ 0.544 }    & \textbf{ 0.544 }    & 0.485    & \textbf{ 0.544 }    & \textbf{ 0.544 }    & \textbf{ 0.544 }    & 0.0 \\
    & F\textsubscript{1}    & 0.548    & 0.544    & 0.552    & \textbf{ 0.561 }    & 0.537    & 0.544    & \textbf{ 0.561 }    & 0.544    & 0.0 \\
    \arrayrulecolor{kit-gray30} \midrule \arrayrulecolor{black}

    \multirow{ 3 }{*}{\rotatebox{90}{ModisDataset}}
    & P.    & 0.212    & 0.212    & 0.224    & 0.4    & 0.333    & 0.145    & 0.435    & 0.244    & \textbf{ 1.0 } \\
    & R.    & \textbf{ 0.268 }    & \textbf{ 0.268 }    & \textbf{ 0.268 }    & 0.244    & 0.024    & \textbf{ 0.268 }    & 0.244    & \textbf{ 0.268 }    & 0.0 \\
    & F\textsubscript{1}    & 0.237    & 0.237    & 0.244    & 0.303    & 0.045    & 0.188    & \textbf{ 0.312 }    & 0.256    & 0.0 \\
    \arrayrulecolor{kit-gray30} \midrule \arrayrulecolor{black}

    \multirow{ 3 }{*}{\rotatebox{90}{WARC}}
    & P.    & 0.394    & 0.38    & 0.377    & 0.437    & 0.534    & 0.377    & 0.438    & 0.379    & \textbf{ 1.0 } \\
    & R.    & \textbf{ 0.699 }    & \textbf{ 0.699 }    & 0.684    & 0.684    & 0.632    & \textbf{ 0.699 }    & 0.669    & 0.691    & 0.0 \\
    & F\textsubscript{1}    & 0.504    & 0.492    & 0.486    & 0.533    & \textbf{ 0.579 }    & 0.49    & 0.529    & 0.49    & 0.0 \\
    \arrayrulecolor{kit-gray30} \midrule \arrayrulecolor{black}

    \multirow{ 3 }{*}{\rotatebox{90}{dronology}}
    & P.    & 0.395    & 0.387    & 0.382    & 0.41    & 0.516    & 0.386    & 0.407    & 0.387    & \textbf{ 1.0 } \\
    & R.    & \textbf{ 0.695 }    & \textbf{ 0.695 }    & 0.682    & 0.686    & 0.577    & \textbf{ 0.695 }    & 0.686    & 0.691    & 0.0 \\
    & F\textsubscript{1}    & 0.504    & 0.498    & 0.489    & 0.514    & \textbf{ 0.545 }    & 0.497    & 0.511    & 0.496    & 0.0 \\
    \arrayrulecolor{kit-gray30} \midrule \arrayrulecolor{black}

    \multirow{ 3 }{*}{\rotatebox{90}{Avg.}}
    & P.    & 0.38    & 0.372    & 0.375    & 0.436    & 0.521    & 0.356    & 0.44    & 0.377    & \textbf{ 1.0 } \\
    & R.    & \textbf{ 0.57 }    & \textbf{ 0.57 }    & 0.564    & 0.56    & 0.446    & \textbf{ 0.57 }    & 0.557    & 0.568    & 0.0 \\
    & F\textsubscript{1}    & 0.449    & 0.443    & 0.441    & \textbf{ 0.474 }    & 0.453    & 0.431    & 0.472    & 0.445    & 0.0 \\
    \arrayrulecolor{kit-gray30} \midrule \arrayrulecolor{black}

    \multirow{ 3 }{*}{\rotatebox{90}{Weighted Avg.}}
    & P.    & 0.397    & 0.388    & 0.387    & 0.434    & 0.527    & 0.38    & 0.434    & 0.39    & \textbf{ 1.0 } \\
    & R.    & \textbf{ 0.637 }    & \textbf{ 0.637 }    & 0.627    & 0.627    & 0.529    & \textbf{ 0.637 }    & 0.623    & 0.633    & 0.0 \\
    & F\textsubscript{1}    & 0.484    & 0.477    & 0.472    & 0.503    & \textbf{ 0.514 }    & 0.471    & 0.501    & 0.476    & 0.0 \\
    \arrayrulecolor{kit-gray30} \midrule \arrayrulecolor{black}

\end{tabularx}
    \renewcommand{\arraystretch}{1}
    \caption{Naive prompt optimization approach prompting the model to optimize the classification prompt}
    \label{tab:naive_optimization}
\end{table}

\newpage
\section{Simple Feedback Optimization}
\label{sec:Evaluation:simple_feedback_optimization}

The following parameters will be used to describe prompts for this section. They can be used as configuration parameters for the feedback optimization process.

\begin{align*}
    mi:=  & \text{maximum iterations}\\
          & \text{the prompt will be optimized $mi$ times} \\
    n  := & \text{feedback sample size}\\  
          & \text{$n$ examples of misclassified \TLs will be provided}
\end{align*}

\begin{prompt}{Feedback Prompt}{feedback_initial}\\
    The current prompt is not performing well in classifying the following trace links. To help you improve the prompt, I will provide examples of misclassified trace links. Please analyze these examples and adjust the prompt accordingly. The examples are as follows: \\
    \{identifier\}\\
    \{source\_type\}: \tripplequote\{source\_content\}\tripplequote\\
    \{target\_type\}: \tripplequote\{target\_content\}\tripplequote\\
    Classification result: \{classification\}
\end{prompt}

\subsection{Optimized Prompt Analysis}

\begin{prompt}{Optimized Prompt n = 3, mi = 1}\\
Question: Here are two parts of software development artifacts.

\{source\_type\}: \tripplequote\{source\_content\}\tripplequote

\{target\_type\}: \tripplequote\{target\_content\}\tripplequote

Consider the following when determining if they are related:\\
1. Look for shared terminology or concepts, such as specific functions, interfaces, or standards mentioned in both requirements.\\
2. Assess whether one requirement describes a feature or functionality that directly supports or complements the other.\\
3. Determine if there is a dependency or a logical connection between the two requirements, such as one requirement ensuring compatibility or compliance that the other relies on.

Are they related?

Answer with \textquotesingle{}yes\textquotesingle{} or \textquotesingle{}no\textquotesingle{}.
\end{prompt}

\begin{prompt}{Optimized Prompt n = 5, mi = 10}\\
Question: Here are two parts of software development artifacts.

\{source\_type\}: \tripplequote\{source\_content\}\tripplequote

\{target\_type\}: \tripplequote\{target\_content\}\tripplequote

Consider the following when determining if they are related:

1. Identify shared terminology or concepts, such as specific functions, interfaces, standards, or types of records mentioned in both requirements. Pay attention to synonyms or closely related terms that may indicate a connection.

2. Evaluate whether one requirement describes a functionality or feature that directly supports, complements, or is necessary for the implementation of the other. Consider if one requirement provides a foundation or prerequisite for the other.

3. Determine if there is a dependency, logical connection, or encapsulation relationship between the two requirements, such as one requirement ensuring compatibility, compliance, or abstraction that the other relies on. Consider if changes in one requirement would impact the other.

\[\dots\]

11. Look for explicit mentions of memory management, remote management, or other specific functionalities that might indicate a relationship, even if not directly related to WARC records.

12. Consider the broader context of the requirements, such as the overall architecture or design principles they support, to identify indirect relationships.

13. Pay attention to the specific phrasing and structure of the requirements, as similar language or structure can indicate a relationship.

14. Consider the intent and purpose behind each requirement, as aligning intents can suggest a relationship even if the specifics differ.

15. Evaluate if the requirements mention or imply the need for compatibility or interoperability, as this can indicate a relationship.

16. Consider if the requirements address scalability, performance, or optimization aspects that might be interrelated.

17. Look for any implicit or explicit dependencies that might not be immediately obvious but are crucial for the system's functionality.

18. Pay attention to the specific types of WARC records mentioned and whether they are addressed in both requirements, as this can indicate a direct relationship.

19. Consider if the requirements mention or imply the need for a single header file or universal interface, as this is a critical aspect in this domain.

20. Reflect on whether the requirements address encapsulation or abstraction, as these are key concepts that often indicate a relationship.

Are they related?

Answer with \textquotesingle{}yes\textquotesingle{} or \textquotesingle{}no\textquotesingle{}.
\end{prompt}
\subsection{Systematic Evaluation Results}

\begin{landscape}
\begin{table}
    \centering
    %TODO: Careful, currently the tabularx width is manually replaced with \hsize in the table to adjust to landscape mode
    \renewcommand{\arraystretch}{1}
    %! Author = ???
%! Date = 2025-08-17

\begin{tabularx}{\textwidth}{p{1.3em}l Z  Z  Z  Z  Z  Z  Z  Z  Z }
    \toprule
    \multirow{2}{1em}[-0.45em]{\rotatebox{90}{Dataset}} & \multirow{2}{*}[-0.9em]{\rotatebox{90}{Metric}}  & \multicolumn{ 3 }{c}{KISS-Original}  & \multicolumn{ 3 }{c}{feedback (mi=10, n=5)}  & \multicolumn{ 3 }{c}{feedback (mi=5, n=3)}                                                                                          \\
        \cmidrule(lr){ 3-5 }
        \cmidrule(lr){ 6-8 }
        \cmidrule(lr){ 9-11 }
    &                            & GPT-4o        & GPT-4o-mini        & Llama 3.1        & GPT-4o        & GPT-4o-mini        & Llama 3.1        & GPT-4o        & GPT-4o-mini        & Llama 3.1           \\
    \midrule
        \multirow{3}{*}{\rotatebox{90}{CCHIT}}
                & P.                     & 0.23  & 0.209  & 0.206  & 0.512  & 0.235  & 0.321  & 0.385  & 0.244  & 0.198  \\
                & R.                     & 0.155     & 0.155     & 0.157     & 0.037     & 0.14     & 0.101     & 0.089     & 0.111     & 0.157     \\
                & F\textsubscript{1}     & 0.185         & 0.178         & 0.178         & 0.07         & 0.175         & 0.153         & 0.144         & 0.152         & 0.175         \\
                \arrayrulecolor{kit-gray30}     \midrule     \arrayrulecolor{black}
        \multirow{3}{*}{\rotatebox{90}{CM1-NASA}}
                & P.                     & 0.349  & 0.337  & 0.33  & 0.333  & 0.367  & 0.364  & 0.345  & 0.372  & 0.33  \\
                & R.                     & 0.644     & 0.644     & 0.644     & 0.644     & 0.644     & 0.444     & 0.644     & 0.644     & 0.644     \\
                & F\textsubscript{1}     & 0.453         & 0.443         & 0.436         & 0.439         & 0.468         & 0.4         & 0.45         & 0.472         & 0.436         \\
                \arrayrulecolor{kit-gray30}     \midrule     \arrayrulecolor{black}
        \multirow{3}{*}{\rotatebox{90}{GANNT}}
                & P.                     & 0.552  & 0.544  & 0.561  & 0.561  & 0.544  & 0.558  & 0.552  & 0.561  & 0.562  \\
                & R.                     & 0.544     & 0.544     & 0.544     & 0.544     & 0.544     & 0.426     & 0.544     & 0.544     & 0.529     \\
                & F\textsubscript{1}     & 0.548         & 0.544         & 0.552         & 0.552         & 0.544         & 0.483         & 0.548         & 0.552         & 0.545         \\
                \arrayrulecolor{kit-gray30}     \midrule     \arrayrulecolor{black}
        \multirow{3}{*}{\rotatebox{90}{ModisDataset}}
                & P.                     & 0.0  & 0.0  & 0.0  & 0.0  & 0.0  & 0.0  & 0.0  & 0.0  & 0.0  \\
                & R.                     & 0.0     & 0.0     & 0.0     & 0.0     & 0.0     & 0.0     & 0.0     & 0.0     & 0.0     \\
                & F\textsubscript{1}     & 0.0         & 0.0         & 0.0         & 0.0         & 0.0         & 0.0         & 0.0         & 0.0         & 0.0         \\
                \arrayrulecolor{kit-gray30}     \midrule     \arrayrulecolor{black}
        \multirow{3}{*}{\rotatebox{90}{WARC}}
                & P.                     & 0.394  & 0.38  & 0.377  & 0.403  & 0.381  & 0.38  & 0.394  & 0.412  & 0.376  \\
                & R.                     & 0.699     & 0.699     & 0.684     & 0.699     & 0.684     & 0.676     & 0.699     & 0.691     & 0.691     \\
                & F\textsubscript{1}     & 0.504         & 0.492         & 0.486         & 0.511         & 0.489         & 0.487         & 0.504         & 0.516         & 0.487         \\
                \arrayrulecolor{kit-gray30}     \midrule     \arrayrulecolor{black}
        \multirow{3}{*}{\rotatebox{90}{dronology}}
                & P.                     & 0.395  & 0.387  & 0.382  & 0.41  & 0.384  & 1.0  & 0.406  & 0.396  & 0.333  \\
                & R.                     & 0.695     & 0.695     & 0.682     & 0.691     & 0.645     & 0.0     & 0.691     & 0.691     & 0.245     \\
                & F\textsubscript{1}     & 0.504         & 0.498         & 0.489         & 0.514         & 0.481         & 0.0         & 0.512         & 0.503         & 0.283         \\
                \arrayrulecolor{kit-gray30}     \midrule     \arrayrulecolor{black}
        \multirow{3}{*}{\rotatebox{90}{Avg.}}
                & P.                     & 0.32  & 0.31  & 0.309  & 0.37  & 0.319  & 0.437  & 0.347  & 0.331  & 0.3  \\
                & R.                     & 0.456     & 0.456     & 0.452     & 0.436     & 0.443     & 0.275     & 0.444     & 0.447     & 0.378     \\
                & F\textsubscript{1}     & 0.366         & 0.359         & 0.357         & 0.348         & 0.359         & 0.254         & 0.36         & 0.366         & 0.321         \\
                \arrayrulecolor{kit-gray30}     \midrule     \arrayrulecolor{black}
\end{tabularx}
    \renewcommand{\arraystretch}{1}
    \caption{Naive prompt optimization approach considering previous misclassified \TLs}
    \label{tab:placeholder}
\end{table}
\end{landscape}

\section{Varying the Optimization Prompt}
In a recent paper by \citewithauthor{zadenoori2025AutomaticPrompt} the authors \Todo{Text needs to be added}.
Their prompt is designed to optimize classification prompts by enhancing the explanations of categories within the prompt.
As it also utilizes feedback from misclassified \TLs, it can be used as an alternative optimization prompt for the naive feedback optimization approach.

They used the following optimization prompt \autoref{prompt:zadenoori_optimization}:
You are required to enhance and clarify the explanations of the categories in the prompt by integrating illustrative examples and information implicitly referenced in the initial context. \\
 The optimized prompt must follow these strict guidelines: \\
 Maintain the Original Steps: The steps in the optimized prompt must remain exactly the same as in the sample prompt; no changes should be made to the steps’ structure or order. \\
 Expand Explanations: Enrich and expand the explanations of each category within the steps, incorporating examples provided.
 Use these examples to enhance understanding and provide clarity, but ensure all content remains within the existing steps and does not extend beyond them. \\
 Incorporate Class Explanations: Specifically, integrate the detailed "Class Explanations" of categories from the first prompt into the optimized prompt.
 For each category, introduce implicit clarifications based on relevant data extracted from the context, keeping all additions within the boundaries of the original steps. \\
 End Strictly After Step 5: The optimized prompt must strictly end after step 5.
 Do not add any additional steps, conclusions, or content beyond this point.

As their optimization prompt is more detailed than the one used in \autoref{sec:Evaluation:simple_feedback_optimization}, it is used as an alternative optimization prompt for the naive optimization approach.
As seen in \autoref{tab:zadenoori_optimization} the results of this optimization approach are quite similar to the ones of \autoref{sec:Evaluation:naive_optimization}.

\newpage
\begin{prompt}{Opimized WARC prompt}{warc_zandoori_optimized}
    Question: Here are two parts of software development artifacts.\\

\{source\_type\}: \tripplequote\{source\_content\}\tripplequote\\

\{target\_type\}: \tripplequote\{target\_content\}\tripplequote\\
Are they related?\\

Answer with 'yes' or 'no'.\\

Class Explanations:\\

1. \textbf{Yes}: This classification is used when the two artifacts are directly related or connected in a meaningful way. For example, if one requirement specifies a feature or functionality that is directly supported or enabled by another requirement, they should be classified as 'Yes'. In the provided examples, if a requirement specifies the need for a universal interface to create WARC records and another requirement details the types of WARC records that can be created through such an interface, they are related and should be classified as 'Yes'. This implies a direct relationship where one requirement's implementation or purpose is explicitly supported or fulfilled by the other. For instance, if "FR 3" specifies providing functions through a universal interface for creating WARC records, and "SRS 7" specifies the types of WARC records that can be created through such an interface, they are directly related. This direct relationship is characterized by one requirement enabling or detailing the implementation of another, ensuring that the functionality or purpose is clearly aligned and supported.\\

2. \textbf{No}: This classification is used when the two artifacts are not directly related or do not have a meaningful connection. For instance, if one requirement discusses the encapsulation of internal functionality and another requirement focuses on ensuring compatibility between versions, they address different aspects and should be classified as 'No'. In the examples provided, if one requirement specifies the need for a single header file and another requirement details the types of WARC records, they are not directly related and should be classified as 'No'. This indicates that the requirements do not share a direct or explicit connection in terms of functionality or purpose, even if they are part of the same broader system or project. For example, "FR 1" requiring a single header file and "SRS 7" detailing WARC record types do not directly relate, as they address different aspects of the system. The lack of a direct relationship is evident when the requirements focus on separate functionalities or objectives, without one directly supporting or fulfilling the other.\\

Incorporate these explanations to ensure accurate classification by identifying direct relationships or lack thereof between the requirements.
\end{prompt}

As their optimization prompt is more detailed than the one used in \autoref{sec:Evaluation:simple_feedback_optimization} , it is used as an alternative optimization prompt for the naive optimization approach.
As seen in \autoref{tab:zadenoori_optimization} the results of this optimization approach are quite similar to the ones of \autoref{sec:Evaluation:naive_optimization}.

\begin{table}
    \centering
    \renewcommand{\arraystretch}{1.4}
    
\begin{tabularx}{\textwidth}{p{1.3em}l Z  Z  Z  Z  Z  Z  Z  Z }
    \toprule
    \multirow{2}{1em}[-0.45em]{\rotatebox{90}{Dataset}} & \multirow{2}{*}[-0.9em]{\rotatebox{90}{Metric}}  & \multicolumn{ 2 }{c}{KISS-Original}  & \multicolumn{ 2 }{c}{feedback i=1, n=3}  & \multicolumn{ 2 }{c}{feedback i=3, n=1}  & \multicolumn{ 2 }{c}{feedback i=5, n=3}                                                                                          \\
    \cmidrule(lr){ 3-4 }
    \cmidrule(lr){ 5-6 }
    \cmidrule(lr){ 7-8 }
    \cmidrule(lr){ 9-10 }
    &                            & GPT-4o        & GPT-4o-mini        & GPT-4o        & GPT-4o-mini        & GPT-4o        & GPT-4o-mini        & GPT-4o        & GPT-4o-mini           \\
    \midrule
    \multirow{ 3 }{*}{\rotatebox{90}{CM1-NASA}}
    & P.    & 0.345    & 0.337    & 0.354    & \textbf{ 0.372 }    & 0.362    & 0.358    & 0.354    & 0.358 \\
    & R.    & \textbf{ 0.644 }    & \textbf{ 0.644 }    & \textbf{ 0.644 }    & \textbf{ 0.644 }    & \textbf{ 0.644 }    & \textbf{ 0.644 }    & \textbf{ 0.644 }    & \textbf{ 0.644 } \\
    & F\textsubscript{1}    & 0.45    & 0.443    & 0.457    & \textbf{ 0.472 }    & 0.464    & 0.46    & 0.457    & 0.46 \\
    \arrayrulecolor{kit-gray30} \midrule \arrayrulecolor{black}

    \multirow{ 3 }{*}{\rotatebox{90}{GANNT}}
    & P.    & 0.552    & 0.544    & 0.552    & 0.544    & \textbf{ 0.569 }    & 0.544    & 0.552    & \textbf{ 0.569 } \\
    & R.    & \textbf{ 0.544 }    & \textbf{ 0.544 }    & \textbf{ 0.544 }    & \textbf{ 0.544 }    & \textbf{ 0.544 }    & \textbf{ 0.544 }    & \textbf{ 0.544 }    & \textbf{ 0.544 } \\
    & F\textsubscript{1}    & 0.548    & 0.544    & 0.548    & 0.544    & \textbf{ 0.556 }    & 0.544    & 0.548    & \textbf{ 0.556 } \\
    \arrayrulecolor{kit-gray30} \midrule \arrayrulecolor{black}

    \multirow{ 3 }{*}{\rotatebox{90}{ModisDataset}}
    & P.    & 0.212    & 0.212    & \textbf{ 0.562 }    & 0.324    & 0.314    & 0.5    & 0.444    & 0.275 \\
    & R.    & \textbf{ 0.268 }    & \textbf{ 0.268 }    & 0.22    & \textbf{ 0.268 }    & \textbf{ 0.268 }    & \textbf{ 0.268 }    & 0.195    & \textbf{ 0.268 } \\
    & F\textsubscript{1}    & 0.237    & 0.237    & 0.316    & 0.293    & 0.289    & \textbf{ 0.349 }    & 0.271    & 0.272 \\
    \arrayrulecolor{kit-gray30} \midrule \arrayrulecolor{black}

    \multirow{ 3 }{*}{\rotatebox{90}{WARC}}
    & P.    & 0.394    & 0.38    & 0.455    & 0.378    & 0.44    & 0.392    & \textbf{ 0.485 }    & 0.383 \\
    & R.    & \textbf{ 0.699 }    & \textbf{ 0.699 }    & \textbf{ 0.699 }    & 0.691    & \textbf{ 0.699 }    & 0.669    & \textbf{ 0.699 }    & 0.684 \\
    & F\textsubscript{1}    & 0.504    & 0.492    & 0.551    & 0.488    & 0.54    & 0.495    & \textbf{ 0.572 }    & 0.491 \\
    \arrayrulecolor{kit-gray30} \midrule \arrayrulecolor{black}

    \multirow{ 3 }{*}{\rotatebox{90}{dronology}}
    & P.    & 0.395    & 0.387    & 0.459    & 0.408    & 0.422    & 0.408    & 0.436    & \textbf{ 0.475 } \\
    & R.    & \textbf{ 0.695 }    & \textbf{ 0.695 }    & 0.686    & 0.686    & 0.691    & 0.686    & 0.686    & 0.614 \\
    & F\textsubscript{1}    & 0.504    & 0.498    & \textbf{ 0.55 }    & 0.512    & 0.524    & 0.512    & 0.534    & 0.536 \\
    \arrayrulecolor{kit-gray30} \midrule \arrayrulecolor{black}

    \multirow{ 3 }{*}{\rotatebox{90}{Avg.}}
    & P.    & 0.38    & 0.372    & \textbf{ 0.476 }    & 0.405    & 0.421    & 0.44    & 0.454    & 0.412 \\
    & R.    & \textbf{ 0.57 }    & \textbf{ 0.57 }    & 0.559    & 0.567    & 0.569    & 0.562    & 0.554    & 0.551 \\
    & F\textsubscript{1}    & 0.449    & 0.443    & \textbf{ 0.484 }    & 0.462    & 0.475    & 0.472    & 0.476    & 0.463 \\
    \arrayrulecolor{kit-gray30} \midrule \arrayrulecolor{black}

    \multirow{ 3 }{*}{\rotatebox{90}{Weighted Avg.}}
    & P.    & 0.397    & 0.388    & \textbf{ 0.469 }    & 0.408    & 0.432    & 0.425    & 0.458    & 0.437 \\
    & R.    & \textbf{ 0.637 }    & \textbf{ 0.637 }    & 0.629    & 0.631    & 0.635    & 0.625    & 0.627    & 0.598 \\
    & F\textsubscript{1}    & 0.484    & 0.477    & \textbf{ 0.523 }    & 0.489    & 0.508    & 0.494    & 0.518    & 0.499 \\
    \arrayrulecolor{kit-gray30} \midrule \arrayrulecolor{black}

\end{tabularx}
    \renewcommand{\arraystretch}{1}
    \caption{Naive prompt optimization approach using the optimization prompt by \citewithauthor{zadenoori2025AutomaticPrompt}}
    \label{tab:zadenoori_optimization}
\end{table}