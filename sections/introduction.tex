%% LaTeX2e class for student theses
%% sections/content.tex
%% 
%% Karlsruhe Institute of Technology
%% Institute for Program Structures and Data Organization
%% Chair for Software Design and Quality (SDQ)
%%
%% Dr.-Ing. Erik Burger
%% burger@kit.edu
%%
%% Version 1.6, 2024-06-07

\chapter{Introduction}
\label{ch:Introduction}
During software development, numerous artifacts are created, ranging from the actual project code to documentation and a multitude of formal and informal diagrams.
\Acl{TLR} aims to link these artifacts across different domains or versions.
Challenges such as inconsistent naming are frequent issues in software projects~\cite{wohlrab2019ImprovingConsistency}.
\TLR approaches need to be able to deal with such hardships.

\autoref{fig:artifact_overview} provides an overview of what these artifacts might look like.
We can see various snippets of a larger project, including requirements, source code, architectural diagrams, and documentation.
For instance, we have the requirement that a \REST \API should be used.
In the documentation, we can read that \REST is used to process images.
The source code shows us that the \method{Image} class is used by the \method{\REST Facade}.
A \TL likely exists between the \method{\REST Facade} and the \REST requirement.
Furthermore, the documentation snippet likely describes the relation between the \method{\REST Facade} and \method{Image}.
So a \TL should exist between them as well.
We can also note naming inconsistencies for the database.
In the architectural diagram, it is referred to as \method{Database}.
At the same time, the source code uses the abbreviated \method{Db} identifier.
Technically, with just the knowledge displayed in \autoref{fig:artifact_overview}, we can not even be certain that \method{Db} is used to abbreviate database.
It might as well be Deutsche Bahn, and a train driver is dependent on a connection to pick up the image from a remote train station. 
\Todo{@advisor Is this fine or too absurd?}

\begin{figure}
    \centering
    \includegraphics[width=0.6\linewidth]{graphics/artifact_overview_Fuchß}
    \caption{Overview of different artifacts during software development by \citewithauthor{fuchss2025LiSSAGeneric}}
    \label{fig:artifact_overview}
\end{figure}

This example illustrates that a multitude of factors need to be considered for \TLR. 
Aside from or rather especially with naming inconsistencies, we need to consider more information to establish \TLs correctly.
\LLMs have made rapid advancements in recent years.
They are becoming increasingly popular for dealing with \TLR tasks and have shown auspicious results so far.
\Todo{This needs proof, sowas sollte belegt werden. du redest hier über trends. LMs wurden schon davor für TLR etc. verwendet :)}
As they are still unable to reason and think on their own~\cite{shojaee2025IllusionThinking}, prompt engineering is required to extract good results.
However, manually determining prompts suited for each specific problem is quite a tedious, time- and labor-intensive task.
The goal is to reduce the effort required for \TLR instead of just shifting the task.

To address this shortcoming, \APO tools can be used.
They will typically use the \LLM itself to determine suitable adjustments or generate new prompts based on a description or training data of the problem~\cite{ramnath2025SystematicSurvey}.

The \LiSSAf proposes an environment for \TLR tasks, including multiple pipeline steps for preprocessing of input data.
They rely on the power of \LLMs to extract traceability links from a variety of different artifacts, also across multiple domains.
My work will contribute an \APO module to this framework.
It can be used to automatically refine a prompt on a dataset and optionally chain the optimization with the regular evaluation pipeline of the \LiSSAf
I am expecting to improve \TLR rates, especially for larger projects in the \RtR domain.

\Todo{Intro sollte länger sein. Figur muss beschrieben sein. ggf beispiel einführen. Problem deutlich hervorheben}