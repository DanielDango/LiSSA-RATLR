%% LaTeX2e class for student theses
%% sections/content.tex
%% 
%% Karlsruhe Institute of Technology
%% Institute for Program Structures and Data Organization
%% Chair for Software Design and Quality (SDQ)
%%
%% Dr.-Ing. Erik Burger
%% burger@kit.edu
%%
%% Version 1.6, 2024-06-07

\chapter{Introduction}
\label{ch:Introduction}

Artificial intelligence is rapidly changing many fields of study.

During software development, many artifacts are created, ranging from the actual project code through documentation to a multitude of formal and informal diagrams. Trace link recovery aims to link these artifacts across different domains or versions. Challenges such as inconsistent naming are frequent issues in software projects \cite{wohlrab2019ImprovingConsistency}. Trace link recovery approaches need to be able to deal with such hardships. 

Large language models (LLM) have made rapid advancements in recent years. As they are still unable to reason and think on their own \cite{shojaee2025IllusionThinking}, prompt engineering is required to extract good results. Manually determining prompts suited for each specific problem is a quite tedious and time intensive task. 

To address this shortcoming, automated prompt engineering tools can be used. They will typically use the LLM itself to determine suitable adjustments or generate new prompts based on a description or training data of the problem \cite{ape}.

\textit{One Page}