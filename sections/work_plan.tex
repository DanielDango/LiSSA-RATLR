\chapter{Work Plan}

\section{Phases}
\subsection{Initial Overview}
\label{phase_initial_overview}

\textbf{What}: To get familiar with the LiSSA framework, I will implement another basic classifier with simple the simple prompting technique tree-of-thought (ToT) \cite{long2023LargeLanguage}. 
Currently, a zero-shot and chain-of-thought classifier are implemented. However, as \citeauthor{long2023LargeLanguage} has shown, tree-of-though style prompts can improve performance compared to chain-of-though approaches in solving logic puzzles. They have tested their approach to ToT prompting with different sized Sudoku puzzles and compared their results against zero-shot prompting as well as chain-of-thought styled one- and few-shot prompting. 


\subsection{Naive iterative Optimization}
\label{phase_naive_iterative}
\textbf{What:} I plan to implement a naive iterative approach to prompt optimization. Many automatic prompt optimization algorithms \citeiterative \textit{to-do: quote more papers using iterative algorithms} depend on an iterative core loop which will be repeated until the optimized prompt performs good enough.

\subsection{Automatic Prompt Optimization Based on Gradient Descent}
\label{phase_gradient_descent}
\textbf{What:} The major implementation of this thesis will be an adaption of the gradient descent based automatic prompt optimization (APO) algorithm by \citewithauthor{pryzant2023AutomaticPrompt}. 
The APO algorithm will be trained and tested on the LiSSA benchmark data \cite{fuchss2022ArDoCoBenchmark}. The optimized prompt will be compared against previous prompts used by \citeauthor{fuchss2025LiSSAGeneric} in the LiSSA project.




\subsection{Evaluation and Buffer}
\label{phase_evaluation}
\textbf{What:} As this work can be seen as an expansion on the recent work of \citewithauthor{fuchss2025LiSSAGeneric} the same metrics will be used to evaluate the different prompts used and optimized in this work. These are the precision, recall, $F_1$-Score and $F_2$-Score. This enables an easy comparison, especially with the manual prompts designed by \citewithauthor{ewald2024RetrievalAugmentedLarge} included in the work of \citeauthor{fuchss2025LiSSAGeneric}.









\section{Artifacts}
This work will yield code to be merged into the LiSSA repository\footnote{https://github.com/ArDoCo/LiSSA-RATLR/}. The implementation of \ref{phase_naive_iterative} and \ref{phase_gradient_descent} will provide a base to add further automatic prompt optimization techniques with similar building blocks.

A written report will be created to document, summarize and evaluate the results of my (automatic) prompt optimization work to retrieve trace links.

\section{Schedule}
\textit{Probably exclude proposal phase}


    \begin{ganttchart}[today=7,today label=Current Week]{1}{25}
      \gantttitle{2025}{25} \\
      %todo: unfortunately not all months have four calendar weeks :)
      %\gantttitlelist{"May","June","July","August","September","October"}{4} \\
      \gantttitlelist{18,...,42}{1} \\
      %\gantttitlelist{1,...,25}{1} \\
      \ganttgroup{Propsoal}{3}{8} \\
      \ganttbar{Litertature research}{1}{4} \\
      \ganttbar{Proposal writing}{4}{7} \\
      \ganttmilestone{Proposal presentation}{8} \\ 
      \ganttgroup{\nameref{phase_initial_overview}}{8}{12} \\
      \ganttbar{LiSSA setup}{8}{9} \\
      \ganttbar{ToT implementation}{10}{10} \\
      \ganttbar{Benchmark setup}{12}{13} \\
      \ganttgroup{Naive optimization}{13}{16} \\
      \ganttbar{Iterative classifier}{13}{13} \\

      \ganttgroup{APO gradient descent}{17}{21} \\
      \ganttbar{Implementation}{17}{19}\\
      \ganttmilestone{Code review}{18} \\ 
      \ganttbar{Thesis writing}{19}{23} \\
      
      \ganttgroup{\nameref{phase_evaluation}}{15}{25} \\
      \ganttbar{Evaluation}{15}{16} 
      \ganttbar{}{20}{21}\\
      \ganttbar{Buffer}{24}{25}
      \ganttbar{}{11}{11}
      \ganttbar{}{14}{14}
      \ganttbar{}{20}{20}\\

      \ganttmilestone{Thesis hand-in}{25}

      \ganttlink{elem9}{elem15}
      \ganttlink{elem11}{elem16}
      \ganttlink{elem7}{elem15}
      \ganttlink{elem3}{elem4}
    \end{ganttchart}

%\section{Risk Management}
%\begin{risks}
%    \item Test
%\end{risks}
