%% LaTeX2e class for student theses
%% sections/abstract_de.tex
%%
%% Karlsruhe Institute of Technology
%% Institute for Program Structures and Data Organization
%% Chair for Software Design and Quality (SDQ)
%%
%% Dr.-Ing. Erik Burger
%% burger@kit.edu
%%
%% Version 1.6, 2024-06-07
%%
%% Translation by Claude Sonnet 4.5

\Abstract

\Ac{TLR} ist eine wichtige Aufgabe im Software Engineering, die dabei hilft, \TLs zwischen verschiedenen Software-Artefakten zu etablieren und zu pflegen.
Traditionelle \TLR-Methoden basieren häufig auf \IR-Techniken, um Kandidaten-Links zu identifizieren.
Neuere Ansätze nutzen \LLMs, um die Genauigkeit und Abrufrate von \TLR zu verbessern.
Um \LLMs jedoch effektiv zu nutzen, ist es entscheidend, geeignete Prompts zu entwerfen.
Dieser Prozess des Prompt Engineerings ist oft manuell und zeitaufwändig und erfordert erhebliche Expertise.

In dieser Arbeit wird ein \APE-Ansatz vorgeschlagen, um den Prompt-Engineering-Prozess für \TLR-Aufgaben im \LiSSAF zu automatisieren.
\Acp{LLM} werden verwendet, um Prompts iterativ zu verfeinern, mit der Fähigkeit, Feedback aus vorherigen Iterationen zu berücksichtigen.
Der Ansatz wird auf fünf Datensätzen aus dem \RtR-Bereich unter Verwendung von drei verschiedenen \LLMs evaluiert.
Als Baseline werden aktuelle Klassifikations-Prompts aus dem \LiSSAF verwendet.

Die Ergebnisse zeigen, dass der vorgeschlagene Ansatz die Leistung von \TLR-Aufgaben leicht verbessern kann.
Die Verbesserungen sind jedoch nicht so signifikant wie zunächst erwartet.
Darüber hinaus wird gezeigt, dass der einzelne, am besten abschneidende optimierte Prompt für einen Datensatz auch auf anderen Datensätzen derselben Domäne gut funktioniert.
Insgesamt trägt diese Arbeit zum Bereich der \TLR bei, indem sie einen Ansatz zum \APE unter Verwendung von \LLMs vorschlägt.
Auch wenn der \APE-Ansatz nicht angewendet wird, können in weiteren Arbeiten die durch diese Arbeit erzielten optimierten Prompts als feste Klassifikations-Prompts verwendet werden.
Dieser Prozess kann mit Evaluierungen im \LiSSAF für kontinuierliche Verbesserung verkettet werden.

Weitere Forschung kann durchgeführt werden, um weitere konfigurierbare Einstellungen zu untersuchen und den \APE-Ansatz auf anderen \TLR-Domänen zu evaluieren.