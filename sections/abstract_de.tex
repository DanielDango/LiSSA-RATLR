%% LaTeX2e class for student theses
%% sections/abstract_de.tex
%%
%% Karlsruhe Institute of Technology
%% Institute for Program Structures and Data Organization
%% Chair for Software Design and Quality (SDQ)
%%
%% Dr.-Ing. Erik Burger
%% burger@kit.edu
%%
%% Version 1.6, 2024-06-07
%%
%% Translated from English to German by Claude Sonnet 4.5


\Abstract

\TLR ist eine wichtige Aufgabe im Software Engineering, die dabei hilft, \TLs zwischen verschiedenen Software-Artefakten herzustellen und zu pflegen.
Traditionelle \TLR-Methoden basieren häufig auf \IR-Techniken, um Kandidatenlinks zu identifizieren.
Neuere Ansätze nutzen \LLMs, um die Genauigkeit und Wiederauffindungsraten von \TLR zu verbessern.
Um \LLMs jedoch effektiv einzusetzen, ist es entscheidend, geeignete Prompts zu gestalten.
Dieser Prozess des Prompt Engineering ist oft manuell und zeitaufwändig und erfordert erhebliche Expertise.
In dieser Arbeit wird ein \APE-Ansatz vorgeschlagen, um den Prompt-Engineering-Prozess für \TLR-Aufgaben im \LiSSAF zu automatisieren.
\LLMs werden verwendet, um Prompts iterativ zu verfeinern, mit der Fähigkeit, Feedback aus vorherigen Iterationen zu berücksichtigen.
Der Ansatz wird auf fünf Datensätzen aus der \RtR-Domäne unter Verwendung von drei verschiedenen \LLMs evaluiert.
Als Baseline werden aktuelle Klassifikations-Prompts aus dem \LiSSAF verwendet.
Die Ergebnisse zeigen, dass der vorgeschlagene Ansatz die Leistung von \TLR-Aufgaben leicht verbessern kann.
Die Verbesserungen sind jedoch nicht so signifikant wie zunächst erwartet.
Darüber hinaus wird gezeigt, dass der einzelne am besten abschneidende optimierte Prompt für einen Datensatz auch auf anderen Datensätzen derselben Domäne gut funktioniert.
Insgesamt leistet diese Arbeit einen Beitrag zum Bereich des \TLR, indem sie einen Ansatz für \APE unter Verwendung von \LLMs vorschlägt.
Selbst wenn der \APE-Ansatz nicht angewendet wird, können in zukünftigen Arbeiten die durch diese Arbeit erzielten optimierten Prompts als feste Klassifikations-Prompts verwendet werden.
Dieser Prozess kann mit Evaluierungen im \LiSSAF verkettet werden, um eine kontinuierliche Verbesserung zu erreichen.
Weitere Forschung kann durchgeführt werden, um weitere konfigurierbare Einstellungen zu untersuchen und den \APE-Ansatz auf anderen \TLR-Domänen zu evaluieren.